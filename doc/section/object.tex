\section{ND-Arrays}
The command \cmdlink{narray} is a TclOO class based on the superclass \cmdlink{::ndlist::ValueContainer}.
It is an object-oriented approach to array manipulation and processing.

\begin{syntax}
\command{narray} new \$nd \$varName <\$value>
\end{syntax}
\begin{syntax}
narray create \$name \$nd \$varName <\$value>
\end{syntax}
\begin{args}
\$nd & Rank of ND-array (e.g. 2D, 2d, or 2 for a matrix). \\
\$varName & Variable to store object name for access and garbage collection. 
Variable names are restricted to word characters and namespace delimiters only.\\
\$value & ND-list value to store in ND-array. Default blank. \\
\$name & Name of object if using ``create'' method.
\end{args}
\subsection{Wrapper Classes for Scalars, Vectors, and Matrices}
For convenience, three wrapper classes have been added: \cmdlink{scalar} for 0D objects, \cmdlink{vector} for 1D objects, and \cmdlink{matrix} for 2D objects. 
These wrapper classes use the \cmdlink{narray} class as a superclass, so all the methods for \cmdlink{narray} apply. 
For brevity, only the ``new'' constructor method is documented here.
\begin{syntax}
\command{scalar} new \$varName <\$value>
\end{syntax}
\begin{syntax}
\command{vector} new \$varName <\$value>
\end{syntax}
\begin{syntax}
\command{matrix} new \$varName <\$value>
\end{syntax}
\begin{args}
\$varName & Variable to store object name for access and garbage collection. 
Variable names are restricted to word characters and namespace delimiters only.\\
\$value & ND-list value to store in ND-array. Default blank. 
\end{args}
\clearpage
\subsection{Value, Rank, Shape, and Size}
The value is accessed by calling the object by itself, the rank is accessed with the method \methodlink[0]{narray}{rank}, and the shape and size are accessed with the methods \methodlink[0]{narray}{shape} and \methodlink[0]{narray}{size}.
\begin{syntax}
\method{narray}{rank}
\end{syntax}
\begin{syntax}
\method{narray}{shape} <\$axis>
\end{syntax}
\begin{syntax}
\method{narray}{size}
\end{syntax}
\begin{args}
\$axis & Axis to get dimension along. Default blank for all axes. \\
\end{args}
\begin{example}{Creating ND-arrays}
\begin{lstlisting}
# Create new ND-arrays
scalar new a {hello world}
vector new b {1 2 3}
matrix new c {{1 2 3} {4 5 6}}
narray new d 3D {{{a b} {c d}} {{e f} {g h}}}
# Print rank and value of ND-arrays
foreach object [list $a $b $c $d] {
    puts [list [$object rank] [$object shape]]
}
\end{lstlisting}
\tcblower
\begin{lstlisting}
0 {}
1 3
2 {2 3}
3 {2 2 2}
\end{lstlisting}
\end{example}

\clearpage
\subsection{Indexing}
The ``\texttt{@}'' operator uses \cmdlink{nget} to access a portion of the ND-array. 
\begin{syntax}
\index{narray methods!@} \$narrayObj @ \$i ...
\end{syntax}
\begin{args}
\$i ... & Index inputs corresponding with rank of ND-array. \\
\end{args}

\begin{example}{Accessing portions of an ND-array}
\begin{lstlisting}
matrix new x {{1 2 3} {4 5 6} {7 8 9}}
puts [$x @ 0 2]
puts [$x @ 0:end-1 {0 2}]
\end{lstlisting}
\tcblower
\begin{lstlisting}
3
{1 3} {4 6}
\end{lstlisting}
\end{example}

\subsection{Copying}
The operator ``\texttt{-{}->}'' copies the ND-array to a new variable, and returns the new object.
If indices are specified, the new ND-array object will have the rank of the indexed range.
\begin{syntax}
\index{narray methods!-{}->} \$narrayObj <@ \$i ...> -{}-> \$varName
\end{syntax}
\begin{args}
\$i ... & Indices to access. Default all. \\
\$varName & Variable to store object name for access and garbage collection. 
Variable names are restricted to word characters and namespace delimiters only.
\end{args}

\begin{example}{Copying a portion of an ND-array}
\begin{lstlisting}
matrix new x {{1 2 3} {4 5 6}}
$x @ 0* : --> y; # Row vector (flattened to 1D)
puts "[$y rank], [$y]"
\end{lstlisting}
\tcblower
\begin{lstlisting}
1, 1 2 3
\end{lstlisting}
\end{example}
\clearpage
\subsection{Evaluation/Mapping}
The command \cmdlink{neval} maps over ND-arrays using \cmdlink{nmap}. 
The command \cmdlink{nexpr} is a special case that passes input through the Tcl \textit{expr} command.
ND-arrays can be referred to with ``\texttt{\$.ref}'', where ``ref'' is the name of the ND-array variable.
Portions of an ND-array can be mapped over with the notation ``\texttt{\$.ref(\$i,...)}''.
Input ND-arrays must all agree in rank or be scalar. 
Additionally, they must have compatible dimensions.
\begin{syntax}
\command{neval} \$body <\$self> <\$rankVar>
\end{syntax}
\begin{syntax}
\command{nexpr} \$expr <\$self> <\$rankVar>
\end{syntax}
\begin{args}
\$body & Script to evaluate, with ``\texttt{\$.ref}'' notation for object references. \\
\$expr & Expression to evaluate, with ``\texttt{\$.ref}'' notation for object references. \\
\$self & Object to refer to with ``\texttt{\$.}''. Default blank. \\
\$rankVar & Variable to store resulting rank in. Default blank.
\end{args}

\begin{example}{Get distance between elements in a vector}
\begin{lstlisting}
vector new x {1 2 4 7 11 16}
puts [nexpr {$.x(1:end) - $.x(0:end-1)}]
\end{lstlisting}
\tcblower
\begin{lstlisting}
1 2 3 4 5
\end{lstlisting}
\end{example}

\begin{example}{Outer product of two vectors}
\begin{lstlisting}
matrix new x {1 2 3}
matrix new y {{4 5 6}}
puts [nexpr {$.x * $.y}]
\end{lstlisting}
\tcblower
\begin{lstlisting}
{4 5 6} {8 10 12} {12 15 18}
\end{lstlisting}
\end{example}

\clearpage
\subsection{Modification}
The assignment operator, ``\texttt{=}'', sets the value of the entire ND-array, or of a specified range.
The math assignment operator, ``\texttt{:=}'', sets the value, passing the input through the \cmdlink{nexpr} command. 
Both assignment operators return the object.

\begin{syntax}
\index{narray methods!=} \$narrayObj <@ \$i ...> = \$value
\end{syntax}
\begin{syntax}
\index{narray methods!:=} \$narrayObj <@ \$i ...> := \$expr
\end{syntax}
\begin{args}
\$i ... & Indices to modify. Default all. \\
\$value & Value to assign. Blank to remove values. \\
\$expr & Expression to evaluate.
\end{args}

If using the math assignment operator, the ND-array or indexed range can be accessed with the alias ``\texttt{\$.}'', and the elements of the array or indexed range can be accessed with ``\texttt{\$.}''.
\begin{syntax}
\$. \$arg ...
\end{syntax}
\begin{args}
\$arg ... & Method arguments for object.
\end{args}

\begin{example}{Element-wise modification of a vector}
\begin{lstlisting}
# Create blank vectors and assign values
[vector new x] = {1 2 3}
[vector new y] = {10 20 30}
# Add one to each element
puts [[$x := {$. + 1}]]
# Double the last element
puts [[$x @ end := {$. * 2}]]
# Element-wise addition of vectors
puts [[$x := {$. + $.y}]]
\end{lstlisting}
\tcblower
\begin{lstlisting}
2 3 4
2 3 8
12 23 38
\end{lstlisting}
\end{example}

\clearpage
\subsection{Removal/Insertion}
The method \methodlink[0]{narray}{remove} removes portions of an ND-array along a specified axis with the command \cmdlink{nremove}, and the method \methodlink[0]{narray}{insert} inserts values into an ND-array at a specified index/axis with the command \cmdlink{ninsert}. 
Both methods modify the object and return the object. 

\begin{syntax}
\method{narray}{remove} \$i <\$axis>
\end{syntax}
\begin{syntax}
\method{narray}{insert} \$i \$sublist <\$axis>
\end{syntax}
\begin{args}
\$i & Indices to remove/insert at. \\
\$sublist & Value to insert. \\
\$axis & Axis to remove/insert at (default 0).
\end{args}

\begin{example}{Removing elements from a vector}
\begin{lstlisting}
vector new vector {1 2 3 4 5 6 7 8}
# Remove all odd numbers
$vector remove [find [nexpr {$.vector % 2}]]
puts [$vector]
\end{lstlisting}
\tcblower
\begin{lstlisting}
2 4 6 8
\end{lstlisting}
\end{example}

\begin{example}{Inserting a column into a matrix}
\begin{lstlisting}
matrix new matrix {{1 2} {3 4} {5 6}}
$matrix insert 1 {A B C} 1
puts [$matrix]
\end{lstlisting}
\tcblower
\begin{lstlisting}
{1 A 2} {3 B 4} {5 C 6}
\end{lstlisting}
\end{example}


\clearpage
\subsection{Map/Reduce}
The method \methodlink[0]{narray}{apply} maps a command over the ND-array with \cmdlink{napply}, and the method \methodlink[0]{narray}{reduce} reduces the ND-array over an axis with \cmdlink{nreduce}. 
Both methods do not modify the object, but rather return values.

\begin{syntax}
\method{narray}{apply} \$command \$arg ...
\end{syntax}
\begin{syntax}
\method{narray}{reduce} \$command <\$axis> \$arg ...
\end{syntax}
\begin{args}
\$command & Command prefix to map over the ND-list object. \\
\$arg ... & Additional arguments to append to command. \\
\$axis & Axis to reduce at (default 0).
\end{args}

\begin{example}{Map a command over a list}
\begin{lstlisting}
vector new text {The quick brown fox jumps over the lazy dog}
puts [$text apply {string length}]; # Print the length of each word
\end{lstlisting}
\tcblower
\begin{lstlisting}
3 5 5 3 5 4 3 4 3
\end{lstlisting}
\end{example}

\begin{example}{Get column statistics of a matrix}
\begin{lstlisting}
matrix new matrix {{1 2 3} {4 5 6} {7 8 9}}
# Convert to double-precision floating point
$matrix = [$matrix apply ::tcl::mathfunc::double]
# Get maximum and minimum of each column
puts [$matrix reduce max]
puts [$matrix reduce min]
\end{lstlisting}
\tcblower
\begin{lstlisting}
7.0 8.0 9.0
1.0 2.0 3.0
\end{lstlisting}
\end{example}

\clearpage
\subsection{Temporary Object Evaluation}
The pipe operator, ``\texttt{|}'', copies the ND-array to a temporary object, and evaluates the method.
Returns the result of the method, or the value of the temporary object.
This operator is useful for converting methods that modify the object to methods that return a modified value.
\begin{syntax}
\index{narray methods!$\vert$} \$narrayObj <@ \$i ...> | \$method \$arg ...
\end{syntax}
\begin{args}
\$i ... & Indices to access. Default all. \\
\$method & Method to evaluate. \\
\$arg ... & Arguments to pass to method.
\end{args}
\begin{example}{Temporary object value}
\begin{lstlisting}
# Create a matrix
matrix new x {{1 2 3} {4 5 6}}
# Print value with first row doubled.
puts [$x | @ 0* : := {$. * 2}]
# Source object was not modified
puts [$x]
\end{lstlisting}
\tcblower
\begin{lstlisting}
{2 4 6} {4 5 6}
{1 2 3} {4 5 6}
\end{lstlisting}
\end{example}
\clearpage
\subsection{Reference Variable Evaluation}
The ampersand operator ``\texttt{\&}'' copies the ND-array value or range to a reference variable, and evaluates a body of script. 
The changes made to the reference variable will be applied to the object, and if the variable is unset, the object will be deleted.
If no indices are specified and the variable is unset in the script, the ND-array object will be destroyed.
Returns the result of the script.

\begin{syntax}
\index{narray methods!\&} \$narrayObj <@ \$i ...> \& \$refName \$body
\end{syntax}
\begin{args}
\$i ... & Indices to access. Default all. \\
\$refName & Variable name to use for reference. \\
\$body & Body to evaluate.
\end{args}

\begin{example}{Appending a vector}
\begin{lstlisting}
# Create a 1D list
vector new x {1 2 3}
# Append the list
$x & ref {lappend ref 4 5 6}
puts [$x]
# Append a subset of the list
$x @ end* & ref {lappend ref 7 8 9}
puts [$x]
\end{lstlisting}
\tcblower
\begin{lstlisting}
1 2 3 4 5 6
1 2 3 4 5 {6 7 8 9}
\end{lstlisting}
\end{example}

\clearpage


\clearpage


