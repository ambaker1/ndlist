\section{N-Dimensional Lists (Tensors)}
A ND-list is defined as a list of equal length (N-1)D-lists, which are defined as equal length (N-2)D-lists, and so on until (N-N)D-lists, which are scalars of arbitrary size.
This definition is flexible, and allows for different interpretations of the same data. 
For example, the list ``1 2 3'' can be interpreted as a scalar with value ``1 2 3'', a vector with values ``1'', ``2'', and ``3'', or a matrix with row vectors ``1'', ``2'', and ``3''. 

The command \cmdlink{ndlist} validates that the input is a valid ND-list. 
If the input value is ``ragged'', as in it has inconsistent dimensions, it will throw an error. 
In general, if a value is a valid for N dimensions, it will also be valid for dimensions 0 to N-1.
All other ND-list commands assume a valid ND-list.
\begin{syntax}
\command{ndlist} \$value <\$nd>
\end{syntax}
\begin{args}
\$value & List to interpret as an ndlist. \\
\$nd & Rank of ND-list (e.g. 2 for matrix) or ``auto'' for auto-rank. Default ``auto''.
\end{args}
\subsection{Shape and Size}
The commands \cmdlink{nshape} and \cmdlink{nsize} return the shape and size of an ND-list, respectively.
The shape is a list of the dimensions, and the size is the product of the shape.
\begin{syntax}
\command{nshape} \$ndlist <\$nd>
\end{syntax}
\begin{syntax}
\command{nsize} \$ndlist <\$nd>
\end{syntax}
\begin{args}
\$ndlist & ND-list to get shape/size of. \\
\$axis & Axis to get dimension along. Default blank for all. \\
\$nd & Rank of ND-list (e.g. 2 for matrix) or ``auto'' for auto-rank. Default ``auto''.
\end{args}
\begin{example}{Getting shape and size of an ND-list}
\begin{lstlisting}
set x {{1 2 3} {4 5 6}}
puts [nshape $x]
puts [nsize $x]
\end{lstlisting}
\tcblower
\begin{lstlisting}
2 3
6
\end{lstlisting}
\end{example}

\clearpage
\subsection{Initialization}
The command \cmdlink{nfull} initializes a valid ND-list of any size filled with a single value.
\begin{syntax}
\command{nfull} \$value \$n ...
\end{syntax}
\begin{args}
\$value & Value to repeat \\
\$n ... & Shape (list of dimensions) of ND-list. 
\end{args}
\begin{example}{Generate ND-list filled with one value}
\begin{lstlisting}
puts [nfull foo 3 2]; # 3x2 matrix filled with "foo"
puts [nfull 0 2 2 2]; # 2x2x2 tensor filled with zeros
\end{lstlisting}
\tcblower
\begin{lstlisting}
{foo foo} {foo foo} {foo foo}
{{0 0} {0 0}} {{0 0} {0 0}}
\end{lstlisting}
\end{example}
The command \cmdlink{nrand} initializes a valid ND-list of any size filled with random values between 0 and 1.
\begin{syntax}
\command{nrand} \$n ...
\end{syntax}
\begin{args}
\$n ... & Shape (list of dimensions) of ND-list. 
\end{args}
\begin{example}{Generate random matrix}
\begin{lstlisting}
expr {srand(0)}; # resets the random number seed (for the example)
puts [nrand 1 2]; # 1x2 matrix filled with random numbers
\end{lstlisting}
\tcblower
\begin{lstlisting}
{0.013469574513598146 0.3831388500440581}
\end{lstlisting}
\end{example}
\clearpage
\subsection{Repeating and Expanding}
The command \cmdlink{nrepeat} repeats portions of an ND-list a specified number of times.
\begin{syntax}
\command{nrepeat} \$ndlist \$n ...
\end{syntax}
\begin{args}
\$value & Value to repeat \\
\$n ... & Repetitions at each level.
\end{args}
\begin{example}{Repeat elements of a matrix}
\begin{lstlisting}
puts [nrepeat {{1 2} {3 4}} 1 2]
\end{lstlisting}
\tcblower
\begin{lstlisting}
{1 2 1 2} {3 4 3 4}
\end{lstlisting}
\end{example}
The command \cmdlink{nexpand} repeats portions of an ND-list to expand to new dimensions.
New dimensions must be divisible by old dimensions.
For example, 1x1, 2x1, 4x1, 1x3, 2x3 and 4x3 are compatible with 4x3.
\begin{syntax}
\command{nexpand} \$ndlist \$n ...
\end{syntax}
\begin{args}
\$ndlist & ND-list to expand. \\
\$n ... & New shape of ND-list. If -1 is used, it keeps that axis the same.
\end{args}
\begin{example}{Expand an ND-list to new dimensions}
\begin{lstlisting}
puts [nexpand {1 2 3} -1 2]
puts [nexpand {{1 2}} 2 4]
\end{lstlisting}
\tcblower
\begin{lstlisting}
{1 1} {2 2} {3 3}
{1 2 1 2} {1 2 1 2}
\end{lstlisting}
\end{example}
\clearpage
\subsection{Padding and Extending}
The command \cmdlink{npad} pads an ND-list along its axes by a specified number of elements.
\begin{syntax}
\command{npad} \$ndlist \$value \$n ...
\end{syntax}
\begin{args}
\$ndlist & ND-list to pad. \\
\$value & Value to pad with. \\
\$n ... & Number of elements to pad.
\end{args}
\begin{example}{Padding an ND-list with zeros}
\begin{lstlisting}
set a {{1 2 3} {4 5 6} {7 8 9}}
puts [npad $a 0 2 1]
\end{lstlisting}
\tcblower
\begin{lstlisting}
{1 2 3 0} {4 5 6 0} {7 8 9 0} {0 0 0 0} {0 0 0 0}
\end{lstlisting}
\end{example}
The command \cmdlink{nextend} extends an ND-list to a new shape by padding.
\begin{syntax}
\command{nextend} \$ndlist \$value \$n ...
\end{syntax}
\begin{args}
\$ndlist & ND-list to extend. \\
\$value & Value to pad with. \\
\$n ... & New shape of ND-list.
\end{args}
\begin{example}{Extending an ND-list to a new shape with a filler value}
\begin{lstlisting}
set a {hello hi hey howdy}
puts [nextend $a world -1 2]
\end{lstlisting}
\tcblower
\begin{lstlisting}
{hello world} {hi world} {hey world} {howdy world}
\end{lstlisting}
\end{example}
\clearpage
\subsection{Flattening and Reshaping}
The command \cmdlink{nflatten} flattens an ND-list to a vector.
\begin{syntax}
\command{nflatten} \$ndlist <\$nd>
\end{syntax}
\begin{args}
\$ndlist & ND-list to flatten. \\
\$nd & Rank of ND-list (e.g. 2 for matrix) or ``auto'' for auto-rank. Default ``auto''.
\end{args}
\begin{example}{Reshape a matrix to a 3D tensor}
\begin{lstlisting}
set x [nflatten {{1 2 3 4} {5 6 7 8}}]
puts [nreshape $x 2 2 2]
\end{lstlisting}
\tcblower
\begin{lstlisting}
{{1 2} {3 4}} {{5 6} {7 8}}
\end{lstlisting}
\end{example}

The command \cmdlink{nreshape} reshapes a vector into specified dimensions.
Sizes must be compatible.
\begin{syntax}
\command{nreshape} \$vector \$n ...
\end{syntax}
\begin{args}
\$vector & Vector (1D-list) to reshape. \\
\$n ... & Shape (list of dimensions) of ND-list. One axis may be dynamic, denoted with a "*".
\end{args}
\begin{example}{Reshape a vector to a matrix with three columns}
\begin{lstlisting}
puts [nreshape {1 2 3 4 5 6} * 3]
\end{lstlisting}
\tcblower
\begin{lstlisting}
{1 2 3} {4 5 6}
\end{lstlisting}
\end{example}


\clearpage

\subsection{Index Notation}
This package provides generalized N-dimensional list access/modification commands, using an index notation parsed by the command \cmdlink{::ndlist::ParseIndex}, which returns the index type and an index list for the type.
\begin{syntax}
\command{::ndlist::ParseIndex} \$n \$input
\end{syntax}
\begin{args}
\$n & Number of elements in list. \\
\$input & Index input. Options are shown below: \\
\quad : & All indices \\
\quad \$start:\$stop & Range of indices (e.g. 0:4 or 1:end-2).\\
\quad \$start:\$step:\$stop & Stepped range of indices (e.g. 0:2:-2 or 2:3:end). \\
\quad \$iList & List of indices (e.g. \{0 end-1 5\} or 3). \\
\quad \$i* & Single index with a asterisk, ``flattens'' the ndlist (e.g. 0* or end-3*). 
\end{args}
Additionally, indices get passed through the \cmdlink{::ndlist::Index2Integer} command, which converts the inputs ``end'', ``end-integer'', ``integer$\pm$integer'' and negative wrap-around indexing (where -1 is equivalent to ``end'') into normal integer indices.
Note that this command will return an error if the index is out of range.
\begin{syntax}
\command{::ndlist::Index2Integer} \$n \$index
\end{syntax}
\begin{args}
\$n & Number of elements in list. \\
\$index & Single index. 
\end{args}

\begin{example}{Index Notation}
\begin{lstlisting}
set n 10
puts [::ndlist::ParseIndex $n :]
puts [::ndlist::ParseIndex $n 1:8]
puts [::ndlist::ParseIndex $n 0:2:6]
puts [::ndlist::ParseIndex $n {0 5 end-1}]
puts [::ndlist::ParseIndex $n end*]
\end{lstlisting}
\tcblower
\begin{lstlisting}
A {}
R {1 8}
L {0 2 4 6}
L {0 5 8}
S 9
\end{lstlisting}
\end{example}
\clearpage
\subsection{Access}
Portions of an ND-list can be accessed with the command \cmdlink{nget}, using the index parser \cmdlink{::ndlist::ParseIndex} for each dimension being indexed.
Note that unlike the Tcl \textit{lindex} and \textit{lrange} commands, \cmdlink{nget} will return an error if the indices are out of range.
\begin{syntax}
\command{nget} \$ndlist \$i ...
\end{syntax}
\begin{args}
\$ndlist & ND-list value. \\
\$i ... & Index inputs, parsed with \cmdlink{::ndlist::ParseIndex}. 
\end{args}
\begin{example}{ND-list access}
\begin{lstlisting}
set A {{1 2 3} {4 5 6} {7 8 9}}
puts [nget $A 0 :]; # get row matrix
puts [nget $A 0* :]; # flatten row matrix to a vector
puts [nget $A 0:1 0:1]; # get matrix subset
puts [nget $A end:0 end:0]; # can have reverse ranges
puts [nget $A {0 0 0} 1*]; # can repeat indices
\end{lstlisting}
\tcblower
\begin{lstlisting}
{1 2 3}
1 2 3
{1 2} {4 5}
{9 8 7} {6 5 4} {3 2 1}
2 2 2
\end{lstlisting}
\end{example}

\clearpage
\subsection{Modification}
A ND-list can be modified by reference with \cmdlink{nset}, and by value with \cmdlink{nreplace}, using the index parser \cmdlink{::ndlist::ParseIndex} for each dimension being indexed.
Note that unlike the Tcl \textit{lset} and \textit{lreplace} commands, the commands \cmdlink{nset} and \cmdlink{nreplace} will return an error if the indices are out of range.
If all the index inputs are ``\texttt{:}'' except for one, and the replacement list is blank, it will delete values along that axis by calling \cmdlink{nremove}.
Otherwise, the replacement ND-list must be expandable to the target index dimensions. 
\begin{syntax}
\command{nset} \$varName \$i ... \$sublist
\end{syntax}
\begin{syntax}
\command{nreplace} \$ndlist \$i ... \$sublist
\end{syntax}
\begin{args}
\$varName & Variable that contains an ND-list. \\
\$ndlist & ND-list to modify. \\
\$i ... & Index inputs, parsed with \cmdlink{::ndlist::ParseIndex}. \\
\$sublist & Replacement list, or blank to delete values.
\end{args}
\begin{example}{Replace range with a single value}
\begin{lstlisting}
puts [nreplace [range 10] 0:2:end 0]
\end{lstlisting}
\tcblower
\begin{lstlisting}
0 1 0 3 0 5 0 7 0 9
\end{lstlisting}
\end{example}
\begin{example}{Swapping matrix rows}
\begin{lstlisting}
set a {{1 2 3} {4 5 6} {7 8 9}}
nset a {1 0} : [nget $a {0 1} :]; # Swap rows and columns (modify by reference)
puts $a
\end{lstlisting}
\tcblower
\begin{lstlisting}
{4 5 6} {1 2 3} {7 8 9}
\end{lstlisting}
\end{example}

\clearpage

\clearpage
\subsection{Removal}
The command \cmdlink{nremove} removes portions of an ND-list at a specified axis.
\begin{syntax}
\command{nremove} \$ndlist \$i <\$axis>
\end{syntax}
\begin{args}
\$ndlist & ND-list to modify. \\
\$i & Index input, parsed with \cmdlink{::ndlist::ParseIndex}. \\
\$axis & Axis to remove at. Default 0.
\end{args}

\begin{example}{Filtering a list by removing elements}
\begin{lstlisting}
set x [range 10]
puts [nremove $x [find $x > 4]]
\end{lstlisting}
\tcblower
\begin{lstlisting}
0 1 2 3 4
\end{lstlisting}
\end{example}

\begin{example}{Deleting a column from a matrix}
\begin{lstlisting}
set a {{1 2 3} {4 5 6} {7 8 9}}
puts [nremove $a 2 1]
\end{lstlisting}
\tcblower
\begin{lstlisting}
{1 2} {4 5} {7 8}
\end{lstlisting}
\end{example}
\clearpage
\subsection{Insertion and Concatenation}
The command \cmdlink{ninsert} inserts an ND-list into another ND-list at a specified index and axis.
The ND-lists must agree in dimension at all other axes.
If ``end'' or ``end-integer'' is used for the index, it will insert after the index. 
Otherwise, it will insert before the index.
The command \cmdlink{ncat} is shorthand for inserting at ``end'', and concatenates two ND-lists.
\begin{syntax}
\command{ninsert} \$ndlist1 \$index \$ndlist2 <\$axis> <\$nd>
\end{syntax}
\begin{syntax}
\command{ncat} \$ndlist1 \$ndlist2 <\$axis> <\$nd> 
\end{syntax}
\begin{args}
\$ndlist1 \$ndlist2 & ND-lists to combine. \\
\$index & Index to insert at. \\
\$axis & Axis to insert/concatenate at (default 0). \\
\$nd & Rank of ND-list (e.g. 2 for matrix) or ``auto'' for auto-rank. Default ``auto''.
\end{args}

\begin{example}{Inserting a column into a matrix}
\begin{lstlisting}
set matrix {{1 2} {3 4} {5 6}}
set column {A B C}
puts [ninsert $matrix 1 $column 1 2]
\end{lstlisting}
\tcblower
\begin{lstlisting}
{1 A 2} {3 B 4} {5 C 6}
\end{lstlisting}
\end{example}
\begin{example}{Concatenate tensors}
\begin{lstlisting}
set x [nreshape {1 2 3 4 5 6 7 8 9} 3 3 1]
set y [nreshape {A B C D E F G H I} 3 3 1]
puts [ncat $x $y 2 3]
\end{lstlisting}
\tcblower
\begin{lstlisting}
{{1 A} {2 B} {3 C}} {{4 D} {5 E} {6 F}} {{7 G} {8 H} {9 I}}
\end{lstlisting}
\end{example}

\clearpage
\subsection{Changing Order of Axes}
The command \cmdlink{nswapaxes} is a general purpose transposing function that swaps the axes of an ND-list.
For simple matrix transposing, the command \cmdlink{transpose} can be used instead.
\begin{syntax}
\command{nswapaxes} \$ndlist \$axis1 \$axis2
\end{syntax}
\begin{args}
\$ndlist & ND-list to manipulate. \\
\$axis1 \$axis2 & Axes to swap.
\end{args}
The command \cmdlink{nmoveaxis} moves a specified source axis to a target position. 
For example, moving axis 0 to position 2 would change ``i,j,k'' to ``j,k,i''.
\begin{syntax}
\command{nmoveaxis} \$ndlist \$source \$target
\end{syntax}
\begin{args}
\$ndlist & ND-list to manipulate. \\
\$source & Source axis. \\
\$target & Target position.
\end{args}
The command \cmdlink{npermute} is more general purpose, and defines a new order for the axes of an ND-list.
For example, the axis list ``1 0 2'' would change ``i,j,k'' to ``j,i,k''. 
\begin{syntax}
\command{npermute} \$ndlist \$axis ...
\end{syntax}
\begin{args}
\$ndlist & ND-list to manipulate. \\
\$axis ... & List of axes defining new order.
\end{args}
\begin{example}{Changing tensor axes}
\begin{lstlisting}
set x {{{1 2} {3 4}} {{5 6} {7 8}}}
set y [nswapaxes $x 0 2]
set z [nmoveaxis $x 0 2]
puts [lindex $x 0 0 1]
puts [lindex $y 1 0 0]
puts [lindex $z 0 1 0]
\end{lstlisting}
\tcblower
\begin{lstlisting}
2
2
2
\end{lstlisting}
\end{example}

\clearpage
\subsection{ND Functional Mapping}
The command \cmdlink{napply} applies a command over each element of an ND-list, and returns the result.
The commands \cmdlink{napply2} maps element-wise over two ND-lists. 
If the input lists have different shapes, they will be expanded to their maximum dimensions with \cmdlink{nexpand} (if compatible).
\begin{syntax}
\command{napply} \$command \$ndlist <\$suffix> <\$nd>
\end{syntax}
\begin{syntax}
\command{napply2} \$command \$ndlist1 \$ndlist2 <\$suffix> <\$nd> 
\end{syntax}
\begin{args}
\$ndlist & ND-list to map over. \\
\$ndlist1 \$ndlist2 & ND-lists to map over, element-wise. \\
\$command & Command prefix to map with. \\
\$suffix & Additional arguments to append to command after ND-list elements. Default blank. \\
\$nd & Rank of ND-list (e.g. 2 for matrix) or ``auto'' for auto-rank. Default ``auto''.
\end{args}
\begin{example}{Chained functional mapping over a matrix}
\begin{lstlisting}
napply puts [napply {format %.2f} [napply expr {{1 2} {3 4}} {+ 1}]]
\end{lstlisting}
\tcblower
\begin{lstlisting}
2.00
3.00
4.00
5.00
\end{lstlisting}
\end{example}
\begin{example}{Format columns of a matrix}
\begin{lstlisting}
set data {{1 2 3} {4 5 6} {7 8 9}}
set formats {{%.1f %.2f %.3f}}
puts [napply2 format $formats $data]
\end{lstlisting}
\tcblower
\begin{lstlisting}
{1.0 2.00 3.000} {4.0 5.00 6.000} {7.0 8.00 9.000}
\end{lstlisting}
\end{example}
\clearpage
\subsection{Reducing an ND-list}
The command \cmdlink{nreduce} combines \cmdlink{nmoveaxis} and \cmdlink{napply} to reduce an axis of an ND-list with a function that reduces a vector to a scalar, like \cmdlink{max} or \cmdlink{sum}.
\begin{syntax}
\command{nreduce} \$command \$ndlist <\$axis> <\$suffix> <\$nd>
\end{syntax}
\begin{args}
\$command & Command prefix to map with. \\
\$ndlist & ND-list to map over. \\
\$axis & Axis to reduce. Default 0. \\
\$suffix & Additional arguments to append to command after ND-list elements. Default blank. \\
\$nd & Rank of ND-list (e.g. 2 for matrix) or ``auto'' for auto-rank. Default ``auto''.
\end{args}
\begin{example}{Matrix row and column statistics}
\begin{lstlisting}
set x {{1 2} {3 4} {5 6} {7 8}}
puts [nreduce max $x]; # max of each column
puts [nreduce max $x 1]; # max of each row
puts [nreduce sum $x]; # sum of each column
puts [nreduce sum $x 1]; # sum of each row
\end{lstlisting}
\tcblower
\begin{lstlisting}
7 8
2 4 6 8
16 20
3 7 11 15
\end{lstlisting}
\end{example}

\clearpage
\subsection{Generalized N-Dimensional Mapping}
The command \cmdlink{nmap} is a general purpose mapping function for N-dimensional lists in Tcl.
If multiple ND-lists are provided for iteration, they must be expandable to their maximum dimensions.
The actual implementation flattens all the ND-lists and calls the Tcl \textit{lmap} command, and then reshapes the result to the target dimensions.
So, if ``continue'' or ``break'' are used in the map body, it will return an error.

\begin{syntax}
\command{nmap} <\$nd> \$varName \$ndlist <\$varName \$ndlist ...> \$body
\end{syntax}
\begin{args}
\$nd & Rank of ND-list (e.g. 2 for matrix) or ``auto'' for auto-rank. Default ``auto''.  \\
\$varName & Variable name to iterate with. \\
\$ndlist & ND-list to iterate over. \\
\$body & Tcl script to evaluate at every loop iteration. 
\end{args}

\begin{example}{Expand and map over matrices}
\begin{lstlisting}
set phrases [nmap 2 greeting {{hello goodbye}} subject {world moon} {
    list $greeting $subject
}]
napply puts $phrases {} 2 
\end{lstlisting}
\tcblower
\begin{lstlisting}
hello world
goodbye world
hello moon
goodbye moon
\end{lstlisting}
\end{example}
\clearpage
\subsection{Loop Index Access}
The iteration indices of \cmdlink{nmap} can be accessed with the commands \cmdlink{i}, \cmdlink{j}, and \cmdlink{k}. 
The commands \cmdlink{j} and \cmdlink{k} are simply shorthand for \cmdlink{i} with axes 1 and 2.
\begin{syntax}
\command{i} <\$axis>
\end{syntax}
\begin{syntax}
\command{j}
\end{syntax}
\begin{syntax}
\command{k}
\end{syntax}
\begin{args}	
\$axis & Dimension to access mapping index at. Default 0. \\
 & If -1, returns the linear index of the loop.
\end{args}

\begin{example}{Finding index tuples that match criteria}
\begin{lstlisting}
set x {{1 2 3} {4 5 6} {7 8 9}}
set indices {}
nmap xi $x {
    if {$xi > 4} {
        lappend indices [list [i] [j]]
    }
}
puts $indices
\end{lstlisting}
\tcblower
\begin{lstlisting}
{1 1} {1 2} {2 0} {2 1} {2 2}
\end{lstlisting}
\end{example}




\clearpage