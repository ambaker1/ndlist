\section{File Import/Export}
The commands \cmdlink{readFile} and \cmdlink{writeFile} perform simple data import/export, while the commands \cmdlink{readMatrix} and \cmdlink{writeMatrix} dynamically convert files to matrix format and matrices to file format.
\begin{syntax}
\command{readFile} <\$option \$value ...> <-newline> \$file
\end{syntax}
\begin{syntax}
\command{readMatrix} <\$option \$value ...> <-newline> \$file
\end{syntax}
\begin{args}
\$option \$value ... & File configuration options, see Tcl \textit{fconfigure} command. \\
-newline & Option to read the final newline if it exists. \\
\$file & File to read data from.
\end{args}
\begin{syntax}
\command{writeFile} <\$option \$value ...> <-nonewline> \$file \$data
\end{syntax}
\begin{syntax}
\command{writeMatrix} <\$option \$value ...> <-nonewline> \$file \$data
\end{syntax}
\begin{args}
\$option \$value ... & File configuration options, see Tcl \textit{fconfigure} command. \\
-nonewline & Option to not write a final newline. \\
\$file & File to write data to. \\
\$data & Data to write to file.
\end{args}
\begin{example}{File import/export}
\begin{lstlisting}
# Export matrix to file (converts to csv)
writeMatrix example.csv {{foo bar} {hello world}}
# Read CSV file
puts [readFile example.csv]
puts [readMatrix example.csv]; # converts from csv to matrix
file delete example.csv
\end{lstlisting}
\tcblower
\begin{lstlisting}
foo,bar
hello,world
{foo bar} {hello world}
\end{lstlisting}
\end{example}

\clearpage
\subsection{Data Conversions}
The commands \cmdlink{mat2txt} and \cmdlink{txt2mat} convert between matrix and space-delimited text, where new-lines separate rows.
Escaping of spaces and newlines is consistent with Tcl rules for valid lists. 
\begin{syntax}
\command{mat2txt} \$mat 
\end{syntax}
\begin{syntax}
\command{txt2mat} \$txt
\end{syntax}
\begin{args}
\$mat & Matrix value. \\
\$txt & Space-delimited values.
\end{args}
The commands \cmdlink{mat2csv} and \cmdlink{csv2mat} convert between matrix and CSV-formatted text, where new lines separate rows.  
Commas and newlines are escaped with quotes, and quotes are escaped with double-quotes. 
\begin{syntax}
\command{mat2csv} \$mat
\end{syntax}
\begin{syntax}
\command{csv2mat} \$csv
\end{syntax}
\begin{args}
\$mat & Matrix value. \\
\$csv & Comma-separated values.
\end{args}
\begin{example}{Data conversions}
\begin{lstlisting}
set matrix {{A B C} {{hello world} foo,bar {"hi"}}}
puts {TXT format:}
puts [mat2txt $matrix]
puts {CSV format:}
puts [mat2csv $matrix]
\end{lstlisting}
\tcblower
\begin{lstlisting}
TXT format:
A B C
{hello world} foo,bar {"hi"}
CSV format:
A,B,C
hello world,"foo,bar","""hi"""
\end{lstlisting}
\end{example}
\clearpage
\subsection{Interface with SQLite Database Tables}
The sqlite3 Tcl package provides access to SQL commands within a Tcl interpreter. 
SQL databases are a powerful way to manipulate and query data, so to aid in using this powerful tool, the following commands were added to the ndlist package: \cmdlink{readTable} and \cmdlink{writeTable}. 
The command \cmdlink{readTable} reads a specified table from a SQL database, returning a matrix. 
The command \cmdlink{writeTable} writes a matrix to a new table (must not exist) in a SQL database. 

\begin{syntax}
\command{readTable} \$db \$table
\end{syntax}
\begin{syntax}
\command{writeTable} \$db \$table \$matrix
\end{syntax}
\begin{args}
\$db & sqlite3 database command \\
\$table & Name of table in database to read/write. \\
\$matrix & Matrix, with first row as headers, to read/write.
\end{args}
\begin{example}{Writing a table to SQLite}
\begin{lstlisting}
# Matrix of data used for table (https://www.tutorialspoint.com/sqlite/company.sql)
set matrix [txt2mat {\
ID          NAME        AGE         ADDRESS     SALARY
1           Paul        32          California  20000.0
2           Allen       25          Texas       15000.0
3           Teddy       23          Norway      20000.0
4           Mark        25          Rich-Mond   65000.0
5           David       27          Texas       85000.0
6           Kim         22          South-Hall  45000.0
7           James       24          Houston     10000.0}]

# Write the data to an SQL table and a Tcl table
package require sqlite3; # required for sqlite3 command
sqlite3 db myDatabase.db; # open SQL database
db eval {DROP TABLE IF EXISTS People}; # delete table if exists
writeTable db People $matrix; # write to SQL table
table new People $matrix; # write to Tcl table

# Example of a query, both with SQL and Tcl table commands
puts [db eval {SELECT NAME FROM People WHERE SALARY > 40000.0;}]
puts [$People mget [$People query {@SALARY > 40000.0}] NAME]
db close; # close SQL database
\end{lstlisting}
\tcblower
\begin{lstlisting}
Mark David Kim
Mark David Kim
\end{lstlisting}
\end{example}
\clearpage
