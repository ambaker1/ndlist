\section{2-Dimensional Lists (Matrices)}
A matrix is a two-dimensional list, or a list of row vectors.
This is consistent with the format used in the Tcllib math::linearalgebra package.
See the example below for how matrices are interpreted.
\begin{equation*}\label{eq:matrix_AB}
A=\begin{bmatrix}
2 & 5 & 1 & 3 \\
4 & 1 & 7 & 9 \\
6 & 8 & 3 & 2 \\
7 & 8 & 1 & 4
\end{bmatrix},\quad
B=\begin{bmatrix}
9 \\ 3 \\ 0 \\ -3
\end{bmatrix},\quad
C = \begin{bmatrix}
3 & 7 & -5 & -2
\end{bmatrix}
\end{equation*}
\begin{example}{Matrices and vectors}
\begin{lstlisting}
# Define matrices, column vectors, and row vectors
set A {{2 5 1 3} {4 1 7 9} {6 8 3 2} {7 8 1 4}}
set B {9 3 0 -3}
set C {{3 7 -5 -2}}
# Print out matrices (join with newline to print out each row)
puts "A ="
puts [join $A \n]
puts "B ="
puts [join $B \n]
puts "C ="
puts [join $C \n]
\end{lstlisting}
\tcblower
\begin{lstlisting}
A =
2 5 1 3
4 1 7 9
6 8 3 2
7 8 1 4
B =
9
3
0
-3
C =
3 7 -5 -2
\end{lstlisting}
\end{example}
\clearpage
\subsection{Combining Matrices}
The commands \cmdlink{stack} and \cmdlink{augment} can be used to combine matrices, row or column-wise.
\begin{syntax}
\command{stack} \$mat1 \$mat2 ...
\end{syntax}
\begin{syntax}
\command{augment} \$mat1 \$mat2 ...
\end{syntax}
\begin{args}
\$mat1 \$mat2 ... & Arbitrary number of matrices to stack/augment (number of columns/rows must match)
\end{args}
The command \cmdlink{block} combines a matrix of matrices into a block matrix.
\begin{syntax}
\command{block} \$matrices
\end{syntax}
\begin{args}
\$matrices & Matrix of matrices.
\end{args}
\begin{example}{Combining matrices}
\begin{lstlisting}
set A [stack {{1 2}} {{3 4}}]
set B [augment {1 2} {3 4}]
set C [block [list [list $A $B] [list $B $A]]]
puts $A
puts $B
puts [join $C \n]; # prints each row on a new line
\end{lstlisting}
\tcblower
\begin{lstlisting}
{1 2} {3 4}
{1 3} {2 4}
1 2 1 3
3 4 2 4
1 3 1 2
2 4 3 4
\end{lstlisting}
\end{example}
\clearpage
\subsection{Matrix Transpose}
The command \cmdlink{transpose} simply swaps the rows and columns of a matrix. 
\begin{syntax}
\command{transpose} \$A
\end{syntax}
\begin{args}
\$A & Matrix to transpose, nxm.
\end{args}
Returns an mxn matrix.
\begin{example}{Transposing a matrix}
\begin{lstlisting}
puts [transpose {{1 2} {3 4}}]
\end{lstlisting}
\tcblower
\begin{lstlisting}
{1 3} {2 4}
\end{lstlisting}
\end{example}
\subsection{Matrix Multiplication}
The command \cmdlink{matmul} performs matrix multiplication for two matrices.
Inner dimensions must match.
\begin{syntax}
\command{matmul} \$A \$B
\end{syntax}
\begin{args}
\$A & Left matrix, nxq. \\
\$B & Right matrix, qxm. 
\end{args}
Returns an nxm matrix (or the corresponding dimensions from additional matrices)
\begin{example}{Multiplying a matrix}
\begin{lstlisting}
puts [matmul {{2 5 1 3} {4 1 7 9} {6 8 3 2} {7 8 1 4}} {9 3 0 -3}]
\end{lstlisting}
\tcblower
\begin{lstlisting}
24 12 72 75
\end{lstlisting}
\end{example}
\clearpage
\subsection{Miscellaneous Linear Algebra Routines}
The command \cmdlink{eye} generates an identity matrix.
\begin{syntax}
\command{eye} \$n
\end{syntax}
\begin{args}
\$n  & Size of identity matrix 
\end{args}

The command \cmdlink{outerprod} takes the outer product of two vectors, $\bm{a} \otimes \bm{b} = \bm{a}\bm{b}^T$.
\begin{syntax}
\command{outerprod} \$a \$b
\end{syntax}
\begin{args}
\$a \$b & Vectors with lengths n and m. Returns a matrix, shape nxm.
\end{args}

The command \cmdlink{kronprod} takes the Kronecker product of two matrices, as shown in \eq\eqref{eq:kronprod}.
\begin{syntax}
\command{kronprod} \$A \$B
\end{syntax}
\begin{args}
\$A \$B & Matrices, shapes nxm and pxq. Returns a matrix, shape (np)x(mq).
\end{args}

\begin{equation}\label{eq:kronprod}
\bm{A} \otimes \bm{B} = \left[\begin{matrix}
a_{11}\bm{B} & ... & a_{1n}\bm{B} \\
\vdots & \ddots & \vdots \\
a_{n1}\bm{B} & ... & a_{nn}\bm{B}
\end{matrix}\right]
\end{equation}
\begin{example}{Outer product and Kronecker product}
\begin{lstlisting}
set A [eye 3]
set B [outerprod {1 2} {3 4}]
set C [kronprod $A $B]
puts [join $C \n]; # prints out each row on a new line
\end{lstlisting}
\tcblower
\begin{lstlisting}
3 4 0 0 0 0
6 8 0 0 0 0
0 0 3 4 0 0
0 0 6 8 0 0
0 0 0 0 3 4
0 0 0 0 6 8
\end{lstlisting}
\end{example}
For more advanced matrix algebra routines, check out the Tcllib math::linearalgebra package.
\clearpage
\subsection{Iteration Tools}
The commands \cmdlink{zip} zips two lists into a list of tuples, and \cmdlink{zip3} zip three lists into a list of triples. 
Lists must be the same length.
\begin{syntax}
\command{zip} \$a \$b
\end{syntax}
\begin{syntax}
\command{zip3} \$a \$b \$c
\end{syntax}
\begin{args}
\$a \$b \$c & Lists to zip together.
\end{args}
\begin{example}{Zipping and unzipping lists}
\begin{lstlisting}
# Zipping
set x [zip {A B C} {1 2 3}]
set y [zip3 {Do Re Mi} {A B C} {1 2 3}]
puts $x
puts $y
# Unzipping (using transpose)
puts [transpose $x]
\end{lstlisting}
\tcblower
\begin{lstlisting}
{A 1} {B 2} {C 3}
{Do A 1} {Re B 2} {Mi C 3}
{A B C} {1 2 3}
\end{lstlisting}
\end{example}
The command \cmdlink{cartprod} computes the Cartesian product of an arbitrary number of vectors, returning a matrix where the columns correspond to the input vectors and the rows correspond to all the combinations of the vector elements.
\begin{syntax}
\command{cartprod} \$a \$b ...
\end{syntax}
\begin{args}
\$a \$b ... & Arbitrary number of vectors to take Cartesian product of.
\end{args}

\begin{example}{Cartesian product}
\begin{lstlisting}
puts [cartprod {A B C} {1 2 3}]
\end{lstlisting}
\tcblower
\begin{lstlisting}
{A 1} {A 2} {A 3} {B 1} {B 2} {B 3} {C 1} {C 2} {C 3}
\end{lstlisting}
\end{example}

\clearpage