\section{1-Dimensional Lists (Vectors)}
Lists are foundational to Tcl, so in addition to providing utilities for ND-lists, this package also provides utilities for working with 1D-lists, or vectors.
\subsection{Range Generator}
The command \cmdlink{range} simply generates a list of integer values. 
This can be used in conjunction with the Tcl \textit{foreach} loop to simplify writing ``for'' loops.
There are two ways of calling this command, as shown below.
\begin{syntax}
\command{range} \$n \\
range \$start \$stop <\$step>
\end{syntax}
\begin{args}
\$n & Number of indices, starting at 0 (e.g. 3 returns 0 1 2). \\
\$start & Starting value. \\
\$stop & Stop value. \\
\$step & Step size. Default 1 or -1, depending on direction of start to stop.
\end{args}
\begin{example}{Integer range generation}
\begin{lstlisting}
puts [range 3]
puts [range 0 2]
puts [range 10 3 -2]
\end{lstlisting}
\tcblower
\begin{lstlisting}
0 1 2
0 1 2
10 8 6 4
\end{lstlisting}
\end{example}
\begin{example}{Simpler for-loop}
\begin{lstlisting}
foreach i [range 3] {
    puts $i
}
\end{lstlisting}
\tcblower
\begin{lstlisting}
0
1
2
\end{lstlisting}
\end{example}
\clearpage
\subsection{Logical Indexing}
The command \cmdlink{find} returns the indices of non-zero elements of a boolean list, or indices of elements that satisfy a given criterion.
Can be used in conjunction with \cmdlink{nget} to perform logical indexing.
\begin{syntax}
\command{find} \$list <\$op \$scalar>
\end{syntax}
\begin{args}
\$list & List of values to compare. \\
\$op & Comparison operator. Default ``!=''. \\
\$scalar & Comparison value. Default 0.
\end{args}
\begin{example}{Filtering a list}
\begin{lstlisting}
set x {0.5 2.3 4.0 2.5 1.6 2.0 1.4 5.6}
puts [nget $x [find $x > 2]]
\end{lstlisting}
\tcblower
\begin{lstlisting}
2.3 4.0 2.5 5.6
\end{lstlisting}
\end{example}
\subsection{Linear Interpolation}
The command \cmdlink{linterp} performs linear 1D interpolation.
Converts input to double.
\begin{syntax}
\command{linterp} \$x \$xList \$yList
\end{syntax}
\begin{args}
\$x & Value to query in \texttt{\$xList} \\
\$xList & List of x points, strictly increasing \\
\$yList & List of y points, same length as \texttt{\$xList}
\end{args}
\begin{example}{Linear interpolation}
\begin{lstlisting}
puts [linterp 2 {1 2 3} {4 5 6}]
puts [linterp 8.2 {0 10 20} {2 -4 5}]
\end{lstlisting}
\tcblower
\begin{lstlisting}
5.0
-2.92
\end{lstlisting}
\end{example}
\clearpage
\subsection{Vector Generation}
The command \cmdlink{linspace} can be used to generate a vector of specified length and equal spacing between two specified values. 
Converts input to double.
\begin{syntax}
\command{linspace} \$n \$start \$stop 
\end{syntax}
\begin{args}
\$n & Number of points \\
\$start & Starting value \\
\$stop & End value
\end{args}
\begin{example}{Linearly spaced vector generation}
\begin{lstlisting}
puts [linspace 5 0 1]
\end{lstlisting}
\tcblower
\begin{lstlisting}
0.0 0.25 0.5 0.75 1.0
\end{lstlisting}
\end{example}
The command \cmdlink{linsteps} generates intermediate values given an increment size and a sequence of targets.
Converts input to double.
\begin{syntax}
\command{linsteps} \$step \$x1 \$x2 ...
\end{syntax}
\begin{args}
\$step & Maximum step size \\
\$x1 \$x2 ... & Targets to hit.
\end{args}
\begin{example}{Intermediate value vector generation}
\begin{lstlisting}
puts [linsteps 0.25 0 1 0]
\end{lstlisting}
\tcblower
\begin{lstlisting}
0.0 0.25 0.5 0.75 1.0 0.75 0.5 0.25 0.0
\end{lstlisting}
\end{example}

\clearpage
\subsection{Functional Mapping}
The command \cmdlink{lapply} simply applies a command over each element of a list, and returns the result.
The command \cmdlink{lapply2} maps element-wise over two equal length lists.
\begin{syntax}
\command{lapply} \$command \$list \$arg ...
\end{syntax}
\begin{syntax}
\command{lapply2} \$command \$list1 \$list2 \$arg ...
\end{syntax}
\begin{args}
\$list & List to map over. \\
\$list1 \$list2 & Lists to map over, element-wise. \\
\$command & Command prefix to map with. \\
\$arg ... & Additional arguments to append to command after list elements. 
\end{args}

\begin{example}{Applying a math function to a list}
\begin{lstlisting}
# Add Tcl math functions to the current namespace path
namespace path [concat [namespace path] ::tcl::mathfunc]
puts [lapply abs {-5 1 2 -2}]
\end{lstlisting}
\tcblower
\begin{lstlisting}
5 1 2 2
\end{lstlisting}
\end{example}

\begin{example}{Mapping over two lists}
\begin{lstlisting}
lapply puts [lapply2 {format "%s %s"} {hello goodbye} {world moon}]
\end{lstlisting}
\tcblower
\begin{lstlisting}
hello world
goodbye moon
\end{lstlisting}
\end{example}

\clearpage
\subsection{List Statistics}
The commands \cmdlink{max}, \cmdlink{min}, \cmdlink{sum}, \cmdlink{product}, \cmdlink{mean}, \cmdlink{median}, \cmdlink{stdev} and \cmdlink{pstdev} compute the maximum, minimum, sum, product, mean, median, sample and population standard deviation of values in a list.
For more advanced statistics, check out the Tcllib math::statistics package.
\begin{syntax}
\command{max} \$list 
\end{syntax}
\begin{syntax}
\command{min} \$list 
\end{syntax}
\begin{syntax}
\command{sum} \$list
\end{syntax}
\begin{syntax}
\command{product} \$list
\end{syntax}
\begin{syntax}
\command{mean} \$list 
\end{syntax}
\begin{syntax}
\command{median} \$list 
\end{syntax}
\begin{syntax}
\command{stdev} \$list
\end{syntax}
\begin{syntax}
\command{pstdev} \$list
\end{syntax}
\begin{args}
\$list & List to compute statistic of. \\
\end{args}
\begin{example}{List Statistics}
\begin{lstlisting}
set list {-5 3 4 0}
foreach stat {max min sum product mean median stdev pstdev} {
    puts [list $stat [$stat $list]]
}
\end{lstlisting}
\tcblower
\begin{lstlisting}
max 4
min -5
sum 2
product 0
mean 0.5
median 1.5
stdev 4.041451884327381
pstdev 3.5
\end{lstlisting}
\end{example}
\clearpage
\subsection{Vector Algebra}
The dot product of two equal length vectors can be computed with \cmdlink{dot}.
The cross product of two vectors of length 3 can be computed with \cmdlink{cross}. 
\begin{syntax}
\command{dot} \$a \$b
\end{syntax}
\begin{syntax}
\command{cross} \$a \$b
\end{syntax}
\begin{args}
\$a & First vector. \\
\$b & Second vector.
\end{args}
The norm, or magnitude, of a vector can be computed with \cmdlink{norm}.
\begin{syntax}
\command{norm} \$a <\$p>
\end{syntax}
\begin{args}
\$a & Vector to compute norm of. \\
\$p & Norm type. 1 is sum of absolute values, 2 is euclidean distance, and Inf is absolute maximum value. Default 2.
\end{args}
\begin{example}{Dot and cross product}
\begin{lstlisting}
set x {1 2 3}
set y {-2 -4 6}
puts [dot $x $y]
puts [cross $x $y]
\end{lstlisting}
\tcblower
\begin{lstlisting}
8
24 -12 0
\end{lstlisting}
\end{example}

For more advanced vector algebra routines, check out the Tcllib math::linearalgebra package.

\clearpage
