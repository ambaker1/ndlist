\documentclass{article}

% Input packages & formatting
% Packages

% Math packages
\usepackage{amsmath} % Extended math functions
\usepackage{amssymb} % Extended math symbols (loads in amsfonts)
\usepackage{bm} % Bold math symbols
\usepackage{mathtools}

% Figure packages
\usepackage{caption} % Caption formatting for university standard
\usepackage{graphicx} % includegraphics command
\usepackage{subcaption} % Subfigures
\usepackage[section]{placeins} % Place floats in section
\usepackage{wrapfig}

% Table packages
\usepackage{booktabs} % Better tables
\usepackage{bigstrut} % Merged table cells
\usepackage{longtable} % Tables which overflow into next page
\usepackage{array}
\usepackage{colortbl} % Color table cells
\usepackage{makecell}
\usepackage{multirow}

% Fonts
\usepackage{lmodern} % Use latin modern rather than computer modern. Better for font encoding.
\usepackage[T1]{fontenc} % Allow text to be searchable in output

% Other packages
\usepackage{appendix} % Appendix environment
\usepackage{nextpage} % Cleartooddpage command
%\usepackage[square,comma,sort,numbers]{natbib} % Reference formatting
\usepackage{setspace} % Line spacing
\usepackage{listings} % Display code with syntax highlighting
\usepackage{upquote} % Vertical quotes in verbatim
\usepackage{xcolor} % Colors
\usepackage{titlesec} % Header spacing
\usepackage{xparse} % for tcolorbox
\usepackage[listings]{tcolorbox} % Colored boxes for highlighting syntax
\tcbuselibrary{breakable}
\tcbuselibrary{skins}
\usepackage{enumitem} % better enumerate/itemize options
\usepackage{fancyhdr}
\usepackage{multicol}
\usepackage{ifthen}
\usepackage{xstring}

% Table of contents
\usepackage{imakeidx} % Index page
\usepackage{tocloft} % Control of table of contents
\usepackage[nottoc]{tocbibind} % Adds bibliography, table of tables, table of figures, to table of contents
\usepackage[bookmarks,linktocpage,hidelinks]{hyperref} % Hyperlinks for sections, figures, etc.

% Formatting
% Page format
\setlength{\oddsidemargin}{0.00in}  % Left side margin for odd numbered pages
\setlength{\evensidemargin}{0.00in} % Right side margin for even numbered pages
\setlength{\topmargin}{0.00in}      % Top margin
\setlength{\headheight}{0.20in}     % Header height
\setlength{\headsep}{0.20in}        % Separation between header and main text
\setlength{\topskip}{0.00in}        % Top skip
\setlength{\textwidth}{6.50in}      % Width of the text
\setlength{\textheight}{8.50in}     % Height of the text
\setlength{\footskip}{0.50in}       % Foot skip
\setlength{\parindent}{0.00in}      % First line indentation
\setlength{\parskip}{6pt}        % Space between two paragraphs

% Captions (figures, tables, etc.)
\setlength{\floatsep}{\parskip}          % Space left between floats.
\setlength{\textfloatsep}{\floatsep}   % Space between last top float
% or first bottom float and the text
\setlength{\intextsep}{\floatsep}      % Space left on top and bottom
% of an in-text float
\setlength{\abovecaptionskip}{0.1in plus 0.25in}  % Space above caption
\setlength{\belowcaptionskip}{0.1in plus 0.25in}  % Space below caption
\setlength{\captionmargin}{0.50in}     % Left/Right margin for caption
\setlength{\abovedisplayskip}{0.00in plus 0.25in} % Space before Math stuff
\setlength{\belowdisplayskip}{0.00in plus 0.25in} % Space after Math stuff
\setlength{\arraycolsep}{0.10in}       % Gap between columns of an array
\setlength{\jot}{0.10in}                % Gap between multiline equations
\setlength{\itemsep}{0.10in}           % Space between successive items

% Counters (no section numbering)
\setcounter{tocdepth}{3}
\setcounter{secnumdepth}{0}

% Spacing
\setstretch{1.5}

\titlespacing*{\section}{0cm}{6pt}{6pt}[0cm]
\titlespacing*{\subsection}{0cm}{6pt}{6pt}[0cm]
\titlespacing*{\subsubsection}{0cm}{6pt}{6pt}[0cm]

\titleformat{\section}
{\sffamily\huge}{}{0pt}{\titlerule\vspace{-0.2cm}}
\titleformat{\subsection}
{\sffamily\itshape\Large}{}{0pt}{}

% Macro for syntax
\newtcolorbox{syntax}{
    size=small,
    sharp corners,
    colframe=black,
    colback=yellow,
    fontupper=\bfseries\ttfamily
}

% Macro for argument table
\newenvironment{args}{
    \begin{tabular}{>{\bfseries\ttfamily}p{0.25\linewidth} p{0.69\linewidth}}
    }{
    \end{tabular}\par
    \vspace{0.5\baselineskip}
}

% Note: Requires packages "listing", "xcolor", and "textcomp"
\lstdefinelanguage{verbatim}{
    basicstyle=\ttfamily\small,
    xleftmargin=9pt,
    xrightmargin=9pt,
    columns=fullflexible,
    keepspaces=true,
    comment=[l]{\#},
    breaklines=true
}

\lstdefinestyle{verbatim}{
    commentstyle=\color{gray},
}

% Example code
\AtBeginDocument{
\newtcolorbox[blend into=listings]{example}[2][]{
    colback=blue!3!white,
    colframe=black,
    colbacktitle=blue!15!white,
    coltitle=black,
    sharp corners,
    enhanced,
    breakable,
    size=small,
    before upper={
        \setstretch{1.0}\lstset{language=verbatim,style=verbatim}\vspace{3pt}\textsf{\textit{Code:}}
    },
    subtitle style={
        colback=blue!20!white,
        fonttitle=\sffamily
    },
    before lower={
        \setstretch{1.0}\lstset{language=verbatim,style=verbatim}\vspace{3pt}\textsf{\textit{Output:}}
    },
    fonttitle=\sffamily,
    title={#2},
    #1
}
}

% Links to sub and subsub commands - optional boolean argument, default true. if false, only displays subcmd.

% Commands (and command ensembles)
\newcommand{\command}[1]{\protect\hypertarget{#1}{#1}\index{#1}}
\newcommand{\subcommand}[2]{\protect\hypertarget{#1 #2}{#1 #2}\index{#1!#2}}
\newcommand{\cmdlink}[1]{\protect\hyperlink{#1}{\textit{#1}}}
\newcommand{\subcmdlink}[3][1]{\protect\hyperlink{#2 #3}{\ifnum#1=1\relax\textit{#2 #3}\else\textit{#3}\fi}}

% Methods (first arg is class)
\newcommand{\method}[2]{\protect\hypertarget{$#1Obj #2}{\$#1Obj #2}\index{#1 methods!#2}}
\newcommand{\methodlink}[3][1]{\protect\hyperlink{$#2Obj #3}{\ifnum#1=1\relax\textit{\$#2Obj #3}\else\textit{#3}\fi}}

% Macros for figure/table names
\newcommand{\fig}{\figurename\ }
\newcommand{\figs}{\figurename s }
\newcommand{\tbl}{\tablename\ }
\newcommand{\tbls}{\tablename s }
\newcommand{\eq}{Eq. }
\newcommand{\eqs}{Eqs. }
\renewcommand{\lstlistingname}{Example}% Listing -> Example
\renewcommand{\lstlistlistingname}{List of \lstlistingname s}% List of Listings -> List of Examples
\newcommand{\ex}{Example }
\newcommand{\exs}{Examples }
\newcommand{\var}[1]{\texttt{\textbf{\$#1}}}

% Header/footer
\renewcommand{\headrulewidth}{0pt}

% Changes to hyperlinks (URLs)
\renewcommand\UrlFont{\color{blue}\rmfamily}

% New column type 
% https://tex.stackexchange.com/questions/75717/how-can-i-mix-itemize-and-tabular-environments
\newcolumntype{L}{>{\labelitemi~~}l<{}}
\newcommand{\version}{0.3.1}

\renewcommand{\cleartooddpage}[1][]{\ignorespaces} % single side
\newcommand{\caret}{$^\wedge$}

\title{\Huge{N-Dimensional Lists (ndlist)}\\\large Version \version}
\author{Alex Baker\\\small\url{https://github.com/ambaker1/ndlist}}
\date{\small\today}
\makeindex[columns=2,title={Command Index}]
\begin{document}
\maketitle
\begin{abstract}
\begin{center}
The ``ndlist'' package is a pure-Tcl package for tensor manipulation and processing.

This package is also a \textcolor{blue}{\href{https://github.com/ambaker1/Tin}{Tin}} package, and can be loaded in as shown below:
\end{center}
\begin{example}{Installing and loading ``ndlist''}
\begin{lstlisting}
package require tin
tin add -auto ndlist https://github.com/ambaker1/ndlist install.tcl
tin import ndlist
\end{lstlisting}
\end{example}
\end{abstract}
\clearpage
\section{1-Dimensional Lists (Vectors)}
Lists are foundational to Tcl, so in addition to providing utilities for ND-lists, this package also provides utilities for working with 1D-lists, or vectors.
\subsection{Range Generator}
The command \cmdlink{range} simply generates a list of integer values. 
This can be used in conjunction with the Tcl \textit{foreach} loop to simplify writing ``for'' loops.
There are two ways of calling this command, as shown below.
\begin{syntax}
\command{range} \$n \\
range \$start \$stop <\$step>
\end{syntax}
\begin{args}
\$n & Number of indices, starting at 0 (e.g. 3 returns 0 1 2). \\
\$start & Starting value. \\
\$stop & Stop value. \\
\$step & Step size. Default 1 or -1, depending on direction of start to stop.
\end{args}
\begin{example}{Integer range generation}
\begin{lstlisting}
puts [range 3]
puts [range 0 2]
puts [range 10 3 -2]
\end{lstlisting}
\tcblower
\begin{lstlisting}
0 1 2
0 1 2
10 8 6 4
\end{lstlisting}
\end{example}
\begin{example}{Simpler for-loop}
\begin{lstlisting}
foreach i [range 3] {
    puts $i
}
\end{lstlisting}
\tcblower
\begin{lstlisting}
0
1
2
\end{lstlisting}
\end{example}
\clearpage
\subsection{Logical Indexing}
The command \cmdlink{find} returns the indices of non-zero elements of a boolean list, or indices of elements that satisfy a given criterion.
Can be used in conjunction with \cmdlink{nget} to perform logical indexing.
\begin{syntax}
\command{find} \$list <\$op \$scalar>
\end{syntax}
\begin{args}
\$list & List of values to compare. \\
\$op & Comparison operator. Default ``!=''. \\
\$scalar & Comparison value. Default 0.
\end{args}
\begin{example}{Filtering a list}
\begin{lstlisting}
set x {0.5 2.3 4.0 2.5 1.6 2.0 1.4 5.6}
puts [nget $x [find $x > 2]]
\end{lstlisting}
\tcblower
\begin{lstlisting}
2.3 4.0 2.5 5.6
\end{lstlisting}
\end{example}
\subsection{Linear Interpolation}
The command \cmdlink{linterp} performs linear 1D interpolation.
Converts input to ``double''.
\begin{syntax}
\command{linterp} \$x \$xList \$yList
\end{syntax}
\begin{args}
\$x & Value to query in \texttt{\$xList} \\
\$xList & List of x points, strictly increasing \\
\$yList & List of y points, same length as \texttt{\$xList}
\end{args}
\begin{example}{Linear interpolation}
\begin{lstlisting}
puts [linterp 2 {1 2 3} {4 5 6}]
puts [linterp 8.2 {0 10 20} {2 -4 5}]
\end{lstlisting}
\tcblower
\begin{lstlisting}
5.0
-2.92
\end{lstlisting}
\end{example}
\clearpage
\subsection{Vector Generation}
The command \cmdlink{linspace} can be used to generate a vector of specified length and equal spacing between two specified values. 
Converts input to ``double''
\begin{syntax}
\command{linspace} \$n \$start \$stop 
\end{syntax}
\begin{args}
\$n & Number of points \\
\$start & Starting value \\
\$stop & End value
\end{args}
\begin{example}{Linearly spaced vector generation}
\begin{lstlisting}
puts [linspace 5 0 1]
\end{lstlisting}
\tcblower
\begin{lstlisting}
0.0 0.25 0.5 0.75 1.0
\end{lstlisting}
\end{example}
The command \cmdlink{linsteps} generates intermediate values given an increment size and a sequence of targets.
Converts input to ``double''.
\begin{syntax}
\command{linsteps} \$step \$x1 \$x2 ...
\end{syntax}
\begin{args}
\$step & Maximum step size \\
\$x1 \$x2 ... & Targets to hit.
\end{args}
\begin{example}{Intermediate value vector generation}
\begin{lstlisting}
puts [linsteps 0.25 0 1 0]
\end{lstlisting}
\tcblower
\begin{lstlisting}
0.0 0.25 0.5 0.75 1.0 0.75 0.5 0.25 0.0
\end{lstlisting}
\end{example}

\clearpage
\subsection{Functional Mapping}
The command \cmdlink{lapply} simply applies a command over each element of a list, and returns the result.
Basic math operators can be mapped over a list with the command \cmdlink{lop}.
\begin{syntax}
\command{lapply} \$command \$list \$arg ...
\end{syntax}
\begin{syntax}
\command{lop} \$list \$op \$arg... 
\end{syntax}
\begin{args}
\$list & List to map over. \\
\$command & Command prefix to map with. \\
\$op & Math operator (see ::tcl::mathop documentation). \\
\$arg ... & Additional arguments to append to command after each list element. 
\end{args}

\begin{example}{Applying a math function to a list}
\begin{lstlisting}
# Add Tcl math functions to the current namespace path
namespace path [concat [namespace path] ::tcl::mathfunc]
puts [lapply abs {-5 1 2 -2}]
\end{lstlisting}
\tcblower
\begin{lstlisting}
5 1 2 2
\end{lstlisting}
\end{example}

\clearpage
\subsection{Mapping Over Two Lists}
The commands \cmdlink{lapply} and \cmdlink{lop} only map over one list.
The commands \cmdlink{lapply2} and \cmdlink{lop2} allow you to map, element-wise, over two lists.
List lengths must be equal. 
\begin{syntax}
\command{lapply2} \$command \$list1 \$list2 \$arg ...
\end{syntax}
\begin{syntax}
\command{lop2} \$list1 \$op \$list2 \$arg... 
\end{syntax}
\begin{args}
\$list1 \$list2 & Lists to map over, element-wise. \\
\$command & Command prefix to map with. \\
\$op & Math operator (see ::tcl::mathop documentation). \\
\$arg ... & Additional arguments to append to command after list elements. \\
\end{args}

\begin{example}{Mapping over two lists}
\begin{lstlisting}
lapply puts [lapply2 {format "%s %s"} {hello goodbye} {world moon}]
\end{lstlisting}
\tcblower
\begin{lstlisting}
hello world
goodbye moon
\end{lstlisting}
\end{example}

\begin{example}{Adding two lists together}
\begin{lstlisting}
puts [lop2 {1 2 3} + {2 3 2}]
\end{lstlisting}
\tcblower
\begin{lstlisting}
3 5 5
\end{lstlisting}
\end{example}

\clearpage
\subsection{List Math}
The Tcl command \textit{lmap} allows you to loop over an arbitrary number of lists in parallel, evaluating a script at each iteration, and collecting the results of each loop iteration into a new list.
The command \cmdlink{lexpr} is an extension of this concept, just calling \textit{lmap} and passing the input through the Tcl \textit{expr} command.

\begin{syntax}
\command{lexpr} \$varList \$list <\$varList \$list ...> \$expr
\end{syntax}
\begin{args}
\$varList ... & List(s) of variables to iterate with. \\
\$list ... & List(s) to iterate over. \\
\$expr & Tcl expression to evaluate at every loop iteration.
\end{args}
\begin{example}{Filtering a list}
\begin{lstlisting}
set numbers [range 10]
set odds [lexpr x $numbers {$x % 2 ? $x : [continue]}]; # only odd numbers
puts $odds
\end{lstlisting}
\tcblower
\begin{lstlisting}
1 3 5 7 9
\end{lstlisting}
\end{example}

\begin{example}{Adding three lists together}
\begin{lstlisting}
set x {1 2 3}
set y {2 9 2}
set z {5 -2 0}
puts [lexpr xi $x yi $y zi $z {$xi + $yi + $zi}]
\end{lstlisting}
\tcblower
\begin{lstlisting}
8 9 5
\end{lstlisting}
\end{example}

\clearpage
\subsection{List Statistics}
The commands \cmdlink{max}, \cmdlink{min}, \cmdlink{sum}, \cmdlink{product}, \cmdlink{mean}, \cmdlink{median}, \cmdlink{stdev} and \cmdlink{pstdev} compute the maximum, minimum, sum, product, mean, median, sample and population standard deviation of values in a list.
For more advanced statistics, check out the Tcllib math::statistics package.
\begin{syntax}
\command{max} \$list 
\end{syntax}
\begin{syntax}
\command{min} \$list 
\end{syntax}
\begin{syntax}
\command{sum} \$list
\end{syntax}
\begin{syntax}
\command{product} \$list
\end{syntax}
\begin{syntax}
\command{mean} \$list 
\end{syntax}
\begin{syntax}
\command{median} \$list 
\end{syntax}
\begin{syntax}
\command{stdev} \$list
\end{syntax}
\begin{syntax}
\command{pstdev} \$list
\end{syntax}
\begin{args}
\$list & List to compute statistic of. \\
\end{args}
\begin{example}{List Statistics}
\begin{lstlisting}
set list {-5 3 4 0}
foreach stat {max min sum product mean median stdev pstdev} {
    puts [list $stat [$stat $list]]
}
\end{lstlisting}
\tcblower
\begin{lstlisting}
max 4
min -5
sum 2
product 0
mean 0.5
median 1.5
stdev 4.041451884327381
pstdev 3.5
\end{lstlisting}
\end{example}
\clearpage
\subsection{Vector Algebra}
The dot product of two equal length vectors can be computed with \cmdlink{dot}.
The cross product of two vectors of length 3 can be computed with \cmdlink{cross}. 
\begin{syntax}
\command{dot} \$a \$b
\end{syntax}
\begin{syntax}
\command{cross} \$a \$b
\end{syntax}
\begin{args}
\$a & First vector. \\
\$b & Second vector.
\end{args}
\begin{example}{Dot and cross product}
\begin{lstlisting}
set x {1 2 3}
set y {-2 -4 6}
puts [dot $x $y]
puts [cross $x $y]
\end{lstlisting}
\tcblower
\begin{lstlisting}
8
24 -12 0
\end{lstlisting}
\end{example}
The norm, or magnitude, of a vector can be computed with \cmdlink{norm}.
\begin{syntax}
\command{norm} \$a <\$p>
\end{syntax}
\begin{args}
\$a & Vector to compute norm of. \\
\$p & Norm type. 1 is sum of absolute values, 2 is euclidean distance, and Inf is absolute maximum value. Default 2.
\end{args}
\begin{example}{Normalizing a vector}
\begin{lstlisting}
set x {3 4}
set x [lop $x / [norm $x]]
puts $x
\end{lstlisting}
\tcblower
\begin{lstlisting}
0.6 0.8
\end{lstlisting}
\end{example}
For more advanced vector algebra routines, check out the Tcllib math::linearalgebra package.

\clearpage
\section{2-Dimensional Lists (Matrices)}
A matrix is a two-dimensional list, or a list of row vectors.
This is consistent with the format used in the Tcllib math::linearalgebra package.
See the example below for how matrices are interpreted.
\begin{equation*}\label{eq:matrix_AB}
A=\begin{bmatrix}
2 & 5 & 1 & 3 \\
4 & 1 & 7 & 9 \\
6 & 8 & 3 & 2 \\
7 & 8 & 1 & 4
\end{bmatrix},\quad
B=\begin{bmatrix}
9 \\ 3 \\ 0 \\ -3
\end{bmatrix},\quad
C = \begin{bmatrix}
3 & 7 & -5 & -2
\end{bmatrix}
\end{equation*}
\begin{example}{Matrices and vectors}
\begin{lstlisting}
set A {{2 5 1 3} {4 1 7 9} {6 8 3 2} {7 8 1 4}}
set B {9 3 0 -3}
set C {{3 7 -5 -2}}
\end{lstlisting}
\end{example}
\subsection{Matrix Transpose}
The command \cmdlink{transpose} simply swaps the rows and columns of a matrix. 
\begin{syntax}
\command{transpose} \$A
\end{syntax}
\begin{args}
\$A & Matrix to transpose, nxm.
\end{args}
Returns an mxn matrix.
\begin{example}{Transposing a matrix}
\begin{lstlisting}
puts [transpose {{1 2} {3 4}}]
\end{lstlisting}
\tcblower
\begin{lstlisting}
{1 3} {2 4}
\end{lstlisting}
\end{example}
\clearpage

\subsection{Stacking and Augmenting Matrices}
The commands \cmdlink{stack} and \cmdlink{augment} can be used to combine matrices, row or column-wise.
Matrices can be combined row-wise or column-wise with the commands \cmdlink{stack} \& \cmdlink{augment}. 
\begin{syntax}
\command{stack} \$mat1 \$mat2 ...
\end{syntax}
\begin{syntax}
\command{augment} \$mat1 \$mat2 ...
\end{syntax}
\begin{args}
\$mat1 \$mat2 ... & Arbitrary number of matrices to stack/augment (number of columns/rows must match)
\end{args}
\begin{example}{Combining matrices}
\begin{lstlisting}
puts [stack {{1 2}} {{3 4}}]
puts [augment {1 2} {3 4}]
\end{lstlisting}
\tcblower
\begin{lstlisting}
{1 2} {3 4}
{1 3} {2 4}
\end{lstlisting}
\end{example}
\subsection{Identity Matrix}
The command \cmdlink{eye} generates an identity matrix of a specified size.
\begin{syntax}
\command{eye} \$n
\end{syntax}
\begin{args}
\$n  & Size of identity matrix 
\end{args}
\begin{example}{Generating an identity matrix}
\begin{lstlisting}
puts [eye 3]
\end{lstlisting}
\tcblower
\begin{lstlisting}
{1 0 0} {0 1 0} {0 0 1}
\end{lstlisting}
\end{example}
\clearpage

\subsection{Matrix Multiplication}
The command \cmdlink{matmul} performs matrix multiplication for two matrices.
Inner dimensions must match.
\begin{syntax}
\command{matmul} \$A \$B
\end{syntax}
\begin{args}
\$A & Left matrix, nxq. \\
\$B & Right matrix, qxm. 
\end{args}
Returns an nxm matrix (or the corresponding dimensions from additional matrices)
\begin{example}{Multiplying a matrix}
\begin{lstlisting}
puts [matmul {{2 5 1 3} {4 1 7 9} {6 8 3 2} {7 8 1 4}} {9 3 0 -3}]
\end{lstlisting}
\tcblower
\begin{lstlisting}
24 12 72 75
\end{lstlisting}
\end{example}

For more advanced matrix algebra routines, check out the Tcllib math::linearalgebra package.

\clearpage
\subsection{Iteration Tools}
The commands \cmdlink{zip} zips two lists into a list of tuples, and \cmdlink{zip3} zip three lists into a list of triples.
Lists must be the same length.
\begin{syntax}
\command{zip} \$a \$b
\end{syntax}
\begin{syntax}
\command{zip3} \$a \$b \$c
\end{syntax}
\begin{args}
\$a \$b \$c & Lists to zip together.
\end{args}
\begin{example}{Zipping lists}
\begin{lstlisting}
puts [zip {A B C} {1 2 3}]
puts [zip3 {Do Re Mi} {A B C} {1 2 3}]
\end{lstlisting}
\tcblower
\begin{lstlisting}
{A 1} {B 2} {C 3}
{Do A 1} {Re B 2} {Mi C 3}
\end{lstlisting}
\end{example}
The command \cmdlink{cartprod} computes the Cartesian product of an arbitrary number of vectors, returning a matrix where the columns correspond to the input vectors and the rows correspond to all the combinations of the vector elements.
\begin{syntax}
\command{cartprod} \$list1 \$list2 ...
\end{syntax}
\begin{args}
\$list1 \$list2 ... & Lists, or vectors, to take Cartesian product of.
\end{args}

\begin{example}{Cartesian product}
\begin{lstlisting}
puts [cartprod {A B C} {1 2 3}]
\end{lstlisting}
\tcblower
\begin{lstlisting}
{A 1} {A 2} {A 3} {B 1} {B 2} {B 3} {C 1} {C 2} {C 3}
\end{lstlisting}
\end{example}

\clearpage
\section{N-Dimensional Lists (Tensors)}
A ND-list is defined as a list of equal length (N-1)D-lists, which are defined as equal length (N-2)D-lists, and so on until (N-N)D-lists, which are scalars of arbitrary size.
This definition is flexible, and allows for different interpretations of the same data. 
For example, the list ``1 2 3'' can be interpreted as a scalar with value ``1 2 3'', a vector with values ``1'', ``2'', and ``3'', or a matrix with row vectors ``1'', ``2'', and ``3''. 

The command \cmdlink{ndlist} validates that the input is a valid ND-list. 
If the input value is ``ragged'', as in it has inconsistent dimensions, it will throw an error. In general, if a value is a valid for N dimensions, it will also be valid for dimensions 0 to N-1.
All other ND-list commands assume a valid ND-list.
\begin{syntax}
\command{ndlist} \$nd \$value
\end{syntax}
\begin{args}
\$nd & Rank of ND-list (e.g. 2D, 2d, or 2 for a matrix).\\
\$value & List to interpret as an ndlist
\end{args}
\subsection{Shape and Size}
The commands \cmdlink{nshape} and \cmdlink{nsize} return the shape and size of an ND-list, respectively.
The shape is a list of the dimensions, and the size is the product of the shape.
\begin{syntax}
\command{nshape} \$nd \$ndlist <\$axis> 
\end{syntax}
\begin{syntax}
\command{nsize} \$nd \$ndlist 
\end{syntax}
\begin{args}
\$nd & Rank of ND-list (e.g. 2D, 2d, or 2 for a matrix).  \\
\$ndlist & ND-list to get dimensions of. \\
\$axis & Axis to get dimension along. Blank for all. 
\end{args}
\begin{example}{Getting shape and size of an ND-list}
\begin{lstlisting}
set A [ndlist 2D {{1 2 3} {4 5 6}}]
puts [nshape 2D $A]
puts [nsize 2D $A]
\end{lstlisting}
\tcblower
\begin{lstlisting}
2 3
6
\end{lstlisting}
\end{example}

\clearpage
\subsection{Initialization}
The command \cmdlink{nfull} initializes a valid ND-list of any size filled with a single value.
\begin{syntax}
\command{nfull} \$value \$n ...
\end{syntax}
\begin{args}
\$value & Value to repeat \\
\$n ... & Shape (list of dimensions) of ND-list. 
\end{args}
\begin{example}{Generate ND-list filled with one value}
\begin{lstlisting}
puts [nfull foo 3 2]; # 3x2 matrix filled with "foo"
puts [nfull 0 2 2 2]; # 2x2x2 tensor filled with zeros
\end{lstlisting}
\tcblower
\begin{lstlisting}
{foo foo} {foo foo} {foo foo}
{{0 0} {0 0}} {{0 0} {0 0}}
\end{lstlisting}
\end{example}
The command \cmdlink{nrand} initializes a valid ND-list of any size filled with random values between 0 and 1.
\begin{syntax}
\command{nrand} \$n ...
\end{syntax}
\begin{args}
\$n ... & Shape (list of dimensions) of ND-list. 
\end{args}
\begin{example}{Generate random matrix}
\begin{lstlisting}
expr {srand(0)}; # resets the random number seed (for the example)
puts [nrand 1 2]; # 1x2 matrix filled with random numbers
\end{lstlisting}
\tcblower
\begin{lstlisting}
{0.013469574513598146 0.3831388500440581}
\end{lstlisting}
\end{example}
\clearpage
\subsection{Repeating and Expanding}
The command \cmdlink{nrepeat} repeats portions of an ND-list a specified number of times.
\begin{syntax}
\command{nrepeat} \$ndlist \$n ...
\end{syntax}
\begin{args}
\$value & Value to repeat \\
\$n ... & Repetitions at each level.
\end{args}
\begin{example}{Repeat elements of a matrix}
\begin{lstlisting}
puts [nrepeat {{1 2} {3 4}} 1 2]
\end{lstlisting}
\tcblower
\begin{lstlisting}
{1 2 1 2} {3 4 3 4}
\end{lstlisting}
\end{example}
The command \cmdlink{nexpand} repeats portions of an ND-list to expand to new dimensions.
New dimensions must be divisible by old dimensions.
For example, 1x1, 2x1, 4x1, 1x3, 2x3 and 4x3 are compatible with 4x3.
\begin{syntax}
\command{nexpand} \$ndlist \$n ...
\end{syntax}
\begin{args}
\$ndlist & ND-list to expand. \\
\$n ... & New dimensions of ND-list.
\end{args}
\begin{example}{Expand an ND-list to new dimensions}
\begin{lstlisting}
puts [nexpand {1 2 3} 3 2]
puts [nexpand {{1 2}} 2 4]
\end{lstlisting}
\tcblower
\begin{lstlisting}
{1 1} {2 2} {3 3}
{1 2 1 2} {1 2 1 2}
\end{lstlisting}
\end{example}
\clearpage
\subsection{Flattening and Reshaping}
The command \cmdlink{nreshape} reshapes a vector into a compatible shape. 
Vector length must equal target ND-list size.
\begin{syntax}
\command{nreshape} \$vector \$n ...
\end{syntax}
\begin{args}
\$vector & Vector (1D-list) to reshape. \\
\$n ... & Shape (list of dimensions) of ND-list. 
\end{args}
\begin{example}{Reshape a vector to a matrix}
\begin{lstlisting}
puts [nreshape {1 2 3 4 5 6} 2 3]
\end{lstlisting}
\tcblower
\begin{lstlisting}
{1 2 3} {4 5 6}
\end{lstlisting}
\end{example}
The inverse is \cmdlink{nflatten}, which flattens an ND-list to a vector, which can be then used with \cmdlink{nreshape}.
\begin{syntax}
\command{nflatten} \$nd \$ndlist
\end{syntax}
\begin{args}
\$nd & Rank of ND-list (e.g. 2D, 2d, or 2 for a matrix).  \\
\$ndlist & ND-list to flatten. 
\end{args}
\begin{example}{Reshape a matrix to a 3D tensor}
\begin{lstlisting}
set x [nflatten 2D {{1 2 3 4} {5 6 7 8}}]
puts [nreshape $x 2 2 2]
\end{lstlisting}
\tcblower
\begin{lstlisting}
{{1 2} {3 4}} {{5 6} {7 8}}
\end{lstlisting}
\end{example}

\clearpage
\subsection{Index Notation}
This package provides generalized N-dimensional list access/modification commands, using an index notation parsed by the command \cmdlink{::ndlist::ParseIndex}, which returns the index type and an index list for the type.
\begin{syntax}
\command{::ndlist::ParseIndex} \$n \$input
\end{syntax}
\begin{args}
\$n & Number of elements in list. \\
\$input & Index input. Options are shown below: \\
\quad : & All indices \\
\quad \$start:\$stop & Range of indices (e.g. 0:4 or 1:end-2).\\
\quad \$start:\$step:\$stop & Stepped range of indices (e.g. 0:2:-2 or 2:3:end). \\
\quad \$iList & List of indices (e.g. \{0 end-1 5\} or 3). \\
\quad \$i* & Single index with a asterisk, ``flattens'' the ndlist (e.g. 0* or end-3*). 
\end{args}
Additionally, indices get passed through the \cmdlink{::ndlist::Index2Integer} command, which converts the inputs ``end'', ``end-integer'', ``integer$\pm$integer'' and negative wrap-around indexing (where -1 is equivalent to ``end'') into normal integer indices.
Note that this command will return an error if the index is out of range.
\begin{syntax}
\command{::ndlist::Index2Integer} \$n \$index
\end{syntax}
\begin{args}
\$n & Number of elements in list. \\
\$index & Single index. 
\end{args}

\begin{example}{Index Notation}
\begin{lstlisting}
set n 10
puts [::ndlist::ParseIndex $n :]
puts [::ndlist::ParseIndex $n 1:8]
puts [::ndlist::ParseIndex $n 0:2:6]
puts [::ndlist::ParseIndex $n {0 5 end-1}]
puts [::ndlist::ParseIndex $n end*]
\end{lstlisting}
\tcblower
\begin{lstlisting}
A {}
R {1 8}
L {0 2 4 6}
L {0 5 8}
S 9
\end{lstlisting}
\end{example}


\clearpage
\subsection{Access}
Portions of an ND-list can be accessed with the command \cmdlink{nget}, using the index parser \cmdlink{::ndlist::ParseIndex} for each dimension being indexed.
Note that unlike the Tcl \textit{lindex} and \textit{lrange} commands, \cmdlink{nget} will return an error if the indices are out of range.
\begin{syntax}
\command{nget} \$ndlist \$i ...
\end{syntax}
\begin{args}
\$ndlist & ND-list value. \\
\$i ... & Index inputs, parsed with \cmdlink{::ndlist::ParseIndex}. 
The number of index arguments determines the interpreted dimensions.
\end{args}
\begin{example}{ND-list access}
\begin{lstlisting}
set A {{1 2 3} {4 5 6} {7 8 9}}
puts [nget $A 0 :]; # get row matrix
puts [nget $A 0* :]; # flatten row matrix to a vector
puts [nget $A 0:1 0:1]; # get matrix subset
puts [nget $A end:0 end:0]; # can have reverse ranges
puts [nget $A {0 0 0} 1*]; # can repeat indices
\end{lstlisting}
\tcblower
\begin{lstlisting}
{1 2 3}
1 2 3
{1 2} {4 5}
{9 8 7} {6 5 4} {3 2 1}
2 2 2
\end{lstlisting}
\end{example}

\clearpage
\subsection{Modification}
A ND-list can be modified by reference with \cmdlink{nset}, and by value with \cmdlink{nreplace}, using the index parser \cmdlink{::ndlist::ParseIndex} for each dimension being indexed.
Note that unlike the Tcl \textit{lset} and \textit{lreplace} commands, the commands \cmdlink{nset} and \cmdlink{nreplace} will return an error if the indices are out of range.
If all the index inputs are ``*'' except for one, and the replacement list is blank, it will delete values along that axis by calling \cmdlink{nremove}.
Otherwise, the replacement ND-list must be expandable to the target index dimensions. 
\begin{syntax}
\command{nset} \$varName \$i ... \$sublist
\end{syntax}
\begin{syntax}
\command{nreplace} \$ndlist \$i ... \$sublist
\end{syntax}
\begin{args}
\$varName & Variable that contains an ND-list (must exist). \\
\$ndlist & ND-list to modify. \\
\$i ... & Index inputs, parsed with \cmdlink{::ndlist::ParseIndex}.
The number of index inputs determines the interpreted dimensions. \\
\$sublist & Replacement list, or blank to delete values.
\end{args}
\begin{example}{Replace range with a single value}
\begin{lstlisting}
puts [nreplace [range 10] 0:2:end 0]
\end{lstlisting}
\tcblower
\begin{lstlisting}
0 1 0 3 0 5 0 7 0 9
\end{lstlisting}
\end{example}
\begin{example}{Swapping matrix rows}
\begin{lstlisting}
set a {{1 2 3} {4 5 6} {7 8 9}}
nset a {1 0} : [nget $a {0 1} :]; # Swap rows and columns (modify by reference)
puts $a
\end{lstlisting}
\tcblower
\begin{lstlisting}
{4 5 6} {1 2 3} {7 8 9}
\end{lstlisting}
\end{example}

\clearpage
\subsection{Removal}
The command \cmdlink{nremove} removes portions of an ND-list at a specified axis.
\begin{syntax}
\command{nremove} \$nd \$ndlist \$i <\$axis>
\end{syntax}
\begin{args}
\$nd & Rank of ND-list (e.g. 2D, 2d, or 2 for a matrix).  \\
\$ndlist & ND-list to modify. \\
\$i & Index input, parsed with \cmdlink{::ndlist::ParseIndex}. \\
\$axis & Axis to remove at. Default 0.
\end{args}

\begin{example}{Filtering a list by removing elements}
\begin{lstlisting}
set x [range 10]
puts [nremove $x [find $x > 4]]
\end{lstlisting}
\tcblower
\begin{lstlisting}
0 1 2 3 4
\end{lstlisting}
\end{example}

\begin{example}{Deleting a column from a matrix}
\begin{lstlisting}
set a {{1 2 3} {4 5 6} {7 8 9}}
puts [nremove $a 2 1]
\end{lstlisting}
\tcblower
\begin{lstlisting}
{1 2} {4 5} {7 8}
\end{lstlisting}
\end{example}

\clearpage
\subsection{Insertion and Concatenation}
The command \cmdlink{ninsert} allows you to insert an ND-list into another ND-list at a specified index and axis, as long as the ND-lists agree in dimension at all other axes.
If ``end'' or ``end-integer'' is used for the index, it will insert after the index. 
Otherwise, it will insert before the index.
\begin{syntax}
\command{ninsert} \$nd \$ndlist1 \$index \$ndlist2 <\$axis>
\end{syntax}
\begin{args}
\$nd & Rank of ND-list (e.g. 2D, 2d, or 2 for a matrix).  \\
\$ndlist1 \$ndlist2 & ND-lists to combine. \\
\$index & Index to insert at. \\
\$axis & Axis to insert at (default 0).
\end{args}
The command \cmdlink{ncat} is shorthand for inserting at ``end'', and concatenates two ND-lists.
\begin{syntax}
\command{ncat} \$nd \$ndlist1 \$ndlist2 <\$axis>
\end{syntax}
\begin{args}
\$nd & Rank of ND-list (e.g. 2D, 2d, or 2 for a matrix).  \\
\$ndlist1 \$ndlist2 & ND-lists to concatenate. \\
\$axis & Axis to concatenate at (default 0).
\end{args}
\begin{example}{Inserting a column into a matrix}
\begin{lstlisting}
set matrix {{1 2} {3 4} {5 6}}
set column {A B C}
puts [ninsert 2D $matrix 1 $column 1]
\end{lstlisting}
\tcblower
\begin{lstlisting}
{1 A 2} {3 B 4} {5 C 6}
\end{lstlisting}
\end{example}
\begin{example}{Concatenate tensors}
\begin{lstlisting}
set x [nreshape {1 2 3 4 5 6 7 8 9} 3 3 1]
set y [nreshape {A B C D E F G H I} 3 3 1]
puts [ncat 3D $x $y 2]
\end{lstlisting}
\tcblower
\begin{lstlisting}
{{1 A} {2 B} {3 C}} {{4 D} {5 E} {6 F}} {{7 G} {8 H} {9 I}}
\end{lstlisting}
\end{example}

\clearpage
\subsection{Changing Order of Axes}
The command \cmdlink{nswapaxes} is a general purpose transposing function that swaps the axes of an ND-list.
For simple matrix transposing, the command \cmdlink{transpose} can be used instead.
\begin{syntax}
\command{nswapaxes} \$ndlist \$axis1 \$axis2
\end{syntax}
\begin{args}
\$ndlist & ND-list to manipulate. \\
\$axis1 \$axis2 & Axes to swap.
\end{args}
The command \cmdlink{nmoveaxis} moves a specified source axis to a target position. 
For example, moving axis 0 to position 2 would change ``i,j,k'' to ``j,k,i''.
\begin{syntax}
\command{nmoveaxis} \$ndlist \$source \$target
\end{syntax}
\begin{args}
\$ndlist & ND-list to manipulate. \\
\$source & Source axis. \\
\$target & Target position.
\end{args}
The command \cmdlink{npermute} is more general purpose, and defines a new order for the axes of an ND-list.
For example, the axis list ``1 0 2'' would change ``i,j,k'' to ``j,i,k''. 
\begin{syntax}
\command{npermute} \$ndlist \$axis ...
\end{syntax}
\begin{args}
\$ndlist & ND-list to manipulate. \\
\$axis ... & List of axes defining new order.
\end{args}
\begin{example}{Changing tensor axes}
\begin{lstlisting}
set x {{{1 2} {3 4}} {{5 6} {7 8}}}
set y [nswapaxes $x 0 2]
set z [nmoveaxis $x 0 2]
puts [lindex $x 0 0 1]
puts [lindex $y 1 0 0]
puts [lindex $z 0 1 0]
\end{lstlisting}
\tcblower
\begin{lstlisting}
2
2
2
\end{lstlisting}
\end{example}

\clearpage
\subsection{ND Functional Mapping}
The command \cmdlink{napply} simply applies a command over each element of an ND-list, and returns the result.
Basic math operators can be mapped over an ND-list with the command \cmdlink{nop}, which is a special case of \cmdlink{napply}, using the ::tcl::mathop namespace.
\begin{syntax}
\command{napply} \$nd \$command \$ndlist \$arg ...
\end{syntax}
\begin{syntax}
\command{nop} \$nd \$ndlist \$op \$arg... 
\end{syntax}
\begin{args}
\$nd & Rank of ND-list (e.g. 2D, 2d, or 2 for a matrix).  \\
\$ndlist & ND-list to map over. \\
\$command & Command prefix to map with. \\
\$op & Math operator (see ::tcl::mathop documentation). \\
\$arg ... & Additional arguments to append to command after ND-list element. 
\end{args}
\begin{example}{Chained functional mapping over a matrix}
\begin{lstlisting}
napply 2D puts [napply 2D {format %.2f} [napply 2D expr {{1 2} {3 4}} + 1]]
\end{lstlisting}
\tcblower
\begin{lstlisting}
2.00
3.00
4.00
5.00
\end{lstlisting}
\end{example}
\begin{example}{Element-wise operations}
\begin{lstlisting}
puts [nop 1D {1 2 3} + 1]
puts [nop 2D {{1 2 3} {4 5 6}} > 2]
\end{lstlisting}
\tcblower
\begin{lstlisting}
2 3 4
{0 0 1} {1 1 1}
\end{lstlisting}
\end{example}

\clearpage
\subsection{Mapping Over Two ND-lists}
The commands \cmdlink{napply} and \cmdlink{nop} only map over one ND-list.
The commands \cmdlink{napply2} and \cmdlink{nop2} allow you to map, element-wise, over two ND-lists. 
If the input lists have different shapes, they will be expanded to their maximum dimensions with \cmdlink{nexpand} (if compatible).
\begin{syntax}
\command{napply2} \$nd \$command \$ndlist1 \$ndlist2 \$arg ...
\end{syntax}
\begin{syntax}
\command{nop2} \$nd \$ndlist1 \$op \$ndlist2 \$arg... 
\end{syntax}
\begin{args}
\$nd & Rank of ND-list (e.g. 2D, 2d, or 2 for a matrix).  \\
\$ndlist1 \$ndlist2 & ND-lists to map over, element-wise. \\
\$command & Command prefix to map with. \\
\$op & Math operator (see ::tcl::mathop documentation). \\
\$arg ... & Additional arguments to append to command after ND-list elements.
\end{args}

\begin{example}{Format columns of a matrix}
\begin{lstlisting}
set data {{1 2 3} {4 5 6} {7 8 9}}
set formats {{%.1f %.2f %.3f}}
puts [napply2 2D format $formats $data]
\end{lstlisting}
\tcblower
\begin{lstlisting}
{1.0 2.00 3.000} {4.0 5.00 6.000} {7.0 8.00 9.000}
\end{lstlisting}
\end{example}
\begin{example}{Adding matrices together}
\begin{lstlisting}
set A {{1 2} {3 4}}
set B {{4 9} {3 1}}
puts [nop2 2D $A + $B]
\end{lstlisting}
\tcblower
\begin{lstlisting}
{5 11} {6 5}
\end{lstlisting}
\end{example}
\clearpage
\subsection{Reducing an ND-list}
The command \cmdlink{nreduce} combines \cmdlink{nmoveaxis} and \cmdlink{napply} to reduce an axis of an ND-list with a function that reduces a vector to a scalar, like \cmdlink{max} or \cmdlink{sum}.
\begin{syntax}
\command{nreduce} \$nd \$command \$ndlist <\$axis> <\$arg ...>
\end{syntax}
\begin{args}
\$nd & Rank of ND-list (e.g. 2D, 2d, or 2 for a matrix).  \\
\$command & Command prefix to map with. \\
\$ndlist & ND-list to map over. \\
\$axis & Axis to reduce. Default 0. \\
\$arg ... & Additional arguments to append to command after ND-list elements.
\end{args}
\begin{example}{Matrix row and column statistics}
\begin{lstlisting}
set x {{1 2} {3 4} {5 6} {7 8}}
puts [nreduce 2D max $x]; # max of each column
puts [nreduce 2D max $x 1]; # max of each row
puts [nreduce 2D sum $x]; # sum of each column
puts [nreduce 2D sum $x 1]; # sum of each row
\end{lstlisting}
\tcblower
\begin{lstlisting}
7 8
2 4 6 8
16 20
3 7 11 15
\end{lstlisting}
\end{example}

\clearpage
\subsection{Generalized N-Dimensional Mapping}
The command \cmdlink{nmap} is a general purpose mapping function for N-dimensional lists in Tcl, and the command \cmdlink{nexpr} a special case for math expressions.
If multiple ND-lists are provided for iteration, they must be expandable to their maximum dimensions.
The actual implementation flattens all the ND-lists and calls the Tcl \textit{lmap} command, and then reshapes the result to the target dimensions.
So, if ``continue'' or ``break'' are used in the map body, it will return an error.

\begin{syntax}
\command{nmap} \$nd \$varName \$ndlist <\$varName \$ndlist ...> \$body
\end{syntax}
\begin{syntax}
\command{nexpr} \$nd \$varName \$ndlist <\$varName \$ndlist ...> \$expr
\end{syntax}
\begin{args}
\$nd & Rank of ND-list (e.g. 2D, 2d, or 2 for a matrix).  \\
\$varName & Variable name to iterate with. \\
\$ndlist & ND-list to iterate over. \\
\$body & Tcl script to evaluate at every loop iteration. \\
\$expr & Tcl expression to evaluate at every loop iteration.
\end{args}

\begin{example}{Expand and map over matrices}
\begin{lstlisting}
set phrases [nmap 2D greeting {{hello goodbye}} subject {world moon} {
    list $greeting $subject
}]
napply 2D puts $phrases
\end{lstlisting}
\tcblower
\begin{lstlisting}
hello world
goodbye world
hello moon
goodbye moon
\end{lstlisting}
\end{example}

\begin{example}{Adding two matrices together, element-wise}
\begin{lstlisting}
set x {{1 2} {3 4}}
set y {{4 1} {3 9}}
set z [nexpr 2D xi $x yi $y {$xi + $yi}]
puts $z
\end{lstlisting}
\tcblower
\begin{lstlisting}
{5 3} {6 13}
\end{lstlisting}
\end{example}
\clearpage
\subsection{Generalized N-Dimensional Looping}
The command \cmdlink{nforeach} is simply a version of \cmdlink{nmap} that returns nothing.
\begin{syntax}
\command{nforeach} \$nd \$varName \$ndlist <\$varName \$ndlist ...> \$body
\end{syntax}
\begin{args}
\$nd & Rank of ND-list (e.g. 2D, 2d, or 2 for a matrix).  \\
\$varName & Variable name to iterate with. \\
\$ndlist & ND-list to iterate over. \\
\$body & Tcl script to evaluate at every loop iteration. 
\end{args}

\subsection{Loop Index Access}
The iteration indices of \cmdlink{nmap}, \cmdlink{nexpr}, or \cmdlink{nforeach} can be accessed with the commands \cmdlink{i}, \cmdlink{j}, and \cmdlink{k}. 
The commands \cmdlink{j} and \cmdlink{k} are simply shorthand for \cmdlink{i} with axes 1 and 2.
\begin{syntax}
\command{i} <\$axis>
\end{syntax}
\begin{syntax}
\command{j}
\end{syntax}
\begin{syntax}
\command{k}
\end{syntax}
\begin{args}	
\$axis & Dimension to access mapping index at. Default 0. \\
 & If -1, returns the linear index of the loop.
\end{args}

\begin{example}{Finding index tuples that match criteria}
\begin{lstlisting}
set x {{1 2 3} {4 5 6} {7 8 9}}
set indices {}
nforeach 2D xi $x {
    if {$xi > 4} {
        lappend indices [list [i] [j]]
    }
}
puts $indices
\end{lstlisting}
\tcblower
\begin{lstlisting}
{1 1} {1 2} {2 0} {2 1} {2 2}
\end{lstlisting}
\end{example}
\clearpage
{\small\printindex}
\end{document}


