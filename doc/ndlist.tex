\documentclass{article}

% Input packages & formatting
% Packages

% Math packages
\usepackage{amsmath} % Extended math functions
\usepackage{amssymb} % Extended math symbols (loads in amsfonts)
\usepackage{bm} % Bold math symbols
\usepackage{mathtools}

% Figure packages
\usepackage{caption} % Caption formatting for university standard
\usepackage{graphicx} % includegraphics command
\usepackage{subcaption} % Subfigures
\usepackage[section]{placeins} % Place floats in section
\usepackage{wrapfig}

% Table packages
\usepackage{booktabs} % Better tables
\usepackage{bigstrut} % Merged table cells
\usepackage{longtable} % Tables which overflow into next page
\usepackage{array}
\usepackage{colortbl} % Color table cells
\usepackage{makecell}
\usepackage{multirow}

% Fonts
\usepackage{lmodern} % Use latin modern rather than computer modern. Better for font encoding.
\usepackage[T1]{fontenc} % Allow text to be searchable in output

% Other packages
\usepackage{appendix} % Appendix environment
\usepackage{nextpage} % Cleartooddpage command
%\usepackage[square,comma,sort,numbers]{natbib} % Reference formatting
\usepackage{setspace} % Line spacing
\usepackage{listings} % Display code with syntax highlighting
\usepackage{upquote} % Vertical quotes in verbatim
\usepackage{xcolor} % Colors
\usepackage{titlesec} % Header spacing
\usepackage{xparse} % for tcolorbox
\usepackage[listings]{tcolorbox} % Colored boxes for highlighting syntax
\tcbuselibrary{breakable}
\tcbuselibrary{skins}
\usepackage{enumitem} % better enumerate/itemize options
\usepackage{fancyhdr}
\usepackage{multicol}
\usepackage{ifthen}
\usepackage{xstring}

% Table of contents
\usepackage{imakeidx} % Index page
\usepackage{tocloft} % Control of table of contents
\usepackage[nottoc]{tocbibind} % Adds bibliography, table of tables, table of figures, to table of contents
\usepackage[bookmarks,linktocpage,hidelinks]{hyperref} % Hyperlinks for sections, figures, etc.

% Formatting
% Page format
\setlength{\oddsidemargin}{0.00in}  % Left side margin for odd numbered pages
\setlength{\evensidemargin}{0.00in} % Right side margin for even numbered pages
\setlength{\topmargin}{0.00in}      % Top margin
\setlength{\headheight}{0.20in}     % Header height
\setlength{\headsep}{0.20in}        % Separation between header and main text
\setlength{\topskip}{0.00in}        % Top skip
\setlength{\textwidth}{6.50in}      % Width of the text
\setlength{\textheight}{8.50in}     % Height of the text
\setlength{\footskip}{0.50in}       % Foot skip
\setlength{\parindent}{0.00in}      % First line indentation
\setlength{\parskip}{6pt}        % Space between two paragraphs

% Captions (figures, tables, etc.)
\setlength{\floatsep}{\parskip}          % Space left between floats.
\setlength{\textfloatsep}{\floatsep}   % Space between last top float
% or first bottom float and the text
\setlength{\intextsep}{\floatsep}      % Space left on top and bottom
% of an in-text float
\setlength{\abovecaptionskip}{0.1in plus 0.25in}  % Space above caption
\setlength{\belowcaptionskip}{0.1in plus 0.25in}  % Space below caption
\setlength{\captionmargin}{0.50in}     % Left/Right margin for caption
\setlength{\abovedisplayskip}{0.00in plus 0.25in} % Space before Math stuff
\setlength{\belowdisplayskip}{0.00in plus 0.25in} % Space after Math stuff
\setlength{\arraycolsep}{0.10in}       % Gap between columns of an array
\setlength{\jot}{0.10in}                % Gap between multiline equations
\setlength{\itemsep}{0.10in}           % Space between successive items

% Counters (no section numbering)
\setcounter{tocdepth}{3}
\setcounter{secnumdepth}{0}

% Spacing
\setstretch{1.5}

\titlespacing*{\section}{0cm}{6pt}{6pt}[0cm]
\titlespacing*{\subsection}{0cm}{6pt}{6pt}[0cm]
\titlespacing*{\subsubsection}{0cm}{6pt}{6pt}[0cm]

\titleformat{\section}
{\sffamily\huge}{}{0pt}{\titlerule\vspace{-0.2cm}}
\titleformat{\subsection}
{\sffamily\itshape\Large}{}{0pt}{}

% Macro for syntax
\newtcolorbox{syntax}{
    size=small,
    sharp corners,
    colframe=black,
    colback=yellow,
    fontupper=\bfseries\ttfamily
}

% Macro for argument table
\newenvironment{args}{
    \begin{tabular}{>{\bfseries\ttfamily}p{0.25\linewidth} p{0.69\linewidth}}
    }{
    \end{tabular}\par
    \vspace{0.5\baselineskip}
}

% Note: Requires packages "listing", "xcolor", and "textcomp"
\lstdefinelanguage{verbatim}{
    basicstyle=\ttfamily\small,
    xleftmargin=9pt,
    xrightmargin=9pt,
    columns=fullflexible,
    keepspaces=true,
    comment=[l]{\#},
    breaklines=true
}

\lstdefinestyle{verbatim}{
    commentstyle=\color{gray},
}

% Example code
\AtBeginDocument{
\newtcolorbox[blend into=listings]{example}[2][]{
    colback=blue!3!white,
    colframe=black,
    colbacktitle=blue!15!white,
    coltitle=black,
    sharp corners,
    enhanced,
    breakable,
    size=small,
    before upper={
        \setstretch{1.0}\lstset{language=verbatim,style=verbatim}\vspace{3pt}\textsf{\textit{Code:}}
    },
    subtitle style={
        colback=blue!20!white,
        fonttitle=\sffamily
    },
    before lower={
        \setstretch{1.0}\lstset{language=verbatim,style=verbatim}\vspace{3pt}\textsf{\textit{Output:}}
    },
    fonttitle=\sffamily,
    title={#2},
    #1
}
}

% Links to sub and subsub commands - optional boolean argument, default true. if false, only displays subcmd.

% Commands (and command ensembles)
\newcommand{\command}[1]{\protect\hypertarget{#1}{#1}\index{#1}}
\newcommand{\subcommand}[2]{\protect\hypertarget{#1 #2}{#1 #2}\index{#1!#2}}
\newcommand{\cmdlink}[1]{\protect\hyperlink{#1}{\textit{#1}}}
\newcommand{\subcmdlink}[3][1]{\protect\hyperlink{#2 #3}{\ifnum#1=1\relax\textit{#2 #3}\else\textit{#3}\fi}}

% Methods (first arg is class)
\newcommand{\method}[2]{\protect\hypertarget{$#1Obj #2}{\$#1Obj #2}\index{#1 methods!#2}}
\newcommand{\methodlink}[3][1]{\protect\hyperlink{$#2Obj #3}{\ifnum#1=1\relax\textit{\$#2Obj #3}\else\textit{#3}\fi}}

% Macros for figure/table names
\newcommand{\fig}{\figurename\ }
\newcommand{\figs}{\figurename s }
\newcommand{\tbl}{\tablename\ }
\newcommand{\tbls}{\tablename s }
\newcommand{\eq}{Eq. }
\newcommand{\eqs}{Eqs. }
\renewcommand{\lstlistingname}{Example}% Listing -> Example
\renewcommand{\lstlistlistingname}{List of \lstlistingname s}% List of Listings -> List of Examples
\newcommand{\ex}{Example }
\newcommand{\exs}{Examples }
\newcommand{\var}[1]{\texttt{\textbf{\$#1}}}

% Header/footer
\renewcommand{\headrulewidth}{0pt}

% Changes to hyperlinks (URLs)
\renewcommand\UrlFont{\color{blue}\rmfamily}

% New column type 
% https://tex.stackexchange.com/questions/75717/how-can-i-mix-itemize-and-tabular-environments
\newcolumntype{L}{>{\labelitemi~~}l<{}}
\newcommand{\version}{0.3.1}

\renewcommand{\cleartooddpage}[1][]{\ignorespaces} % single side
\newcommand{\caret}{$^\wedge$}

\title{\Huge{N-Dimensional Lists (ndlist)}\\\large Version \version}
\author{Alex Baker\\\small\url{https://github.com/ambaker1/ndlist}}
\date{\small\today}
\makeindex[columns=2,title={Command Index}]
\begin{document}
\maketitle
\begin{abstract}
\begin{center}
The ``ndlist'' module provides tools for vector, matrix, and tensor manipulation and processing, where vectors are represented by Tcl lists, and matrices are represented by nested Tcl lists, and higher dimension lists represented by additional levels of nesting. 
Additionally, this package provides the ``ndlist'' object variable type, using the object variable framework provided by the \textcolor{blue}{\href{https://github.com/ambaker1/vutil}{vutil}} package.
\end{center}
\end{abstract}
\clearpage
\section{N-Dimensional Lists}
A ND list is defined as a list of equal length (N-1)D lists, which are defined as equal length (N-2)D lists, and so on until (N-N)D lists, which are scalars of arbitrary size.
For example, a matrix is a 2D list, or a list of equal length row vectors (1D), which contain arbitrary scalar values, as shown below:
\begin{equation*}\label{eq:matrix_AB}
A=\begin{bmatrix}
2 & 5 & 1 & 3 \\
4 & 1 & 7 & 9 \\
6 & 8 & 3 & 2 \\
7 & 8 & 1 & 4
\end{bmatrix},\quad
B=\begin{bmatrix}
9 \\ 3 \\ 0 \\ -3
\end{bmatrix},\quad
C = \begin{bmatrix}
3 & 7 & -5 & -2
\end{bmatrix}
\end{equation*}
\begin{example}[label=ex:matrix_AB]{Defining matrices in Tcl}
\begin{lstlisting}
set A {{2 5 1 3} {4 1 7 9} {6 8 3 2} {7 8 1 4}}
set B {9 3 0 -3}
set C {{3 7 -5 -2}}
\end{lstlisting}
\end{example}

This definition is flexible, and allows for different interpretations of the same data. For example, the list ``1 2 3'' can be interpreted as a scalar with value ``1 2 3'', a vector with values ``1'', ``2'', and ``3'', or a matrix with row vectors ``1'', ``2'', and ``3''.

\clearpage
\subsection{Initialization}
The command \cmdlink{nrepeat} can be used to initialize a valid ND list of any size.
\begin{syntax}
\command{nrepeat} \$shape \$value
\end{syntax}
\begin{args}
\$shape & Shape (list of dimensions) of ND list. \\
\$value & Value to repeat.
\end{args}
\begin{example}{Create nested ND list with one value}
\begin{lstlisting}
puts [nrepeat {1 2 3} 0]
\end{lstlisting}
\tcblower
\begin{lstlisting}
{{0 0 0} {0 0 0}}
\end{lstlisting}
\end{example}

The command \cmdlink{ndlist} initializes an ND list from a value input, expanding if necessary to match maximum dimensions.
If the input value is ``ragged'', as in it has inconsistent dimensions, it will be expanded to match the maximum dimensions, filled in with blanks.
\begin{syntax}
\command{ndlist} \$nd \$value
\end{syntax}
\begin{args}
\$nd & Dimensionality of ND list (e.g. 2D for a matrix).  \\
\$value & List to interpret as an ndlist
\end{args}
\begin{example}{Expand ragged ND list}
\begin{lstlisting}
puts [ndlist 2D {1 {2 3}}]
\end{lstlisting}
\tcblower
\begin{lstlisting}
{{1 {}} {2 3}}
\end{lstlisting}
\end{example}

\clearpage

\subsection{ND List Access}
Portions of an ND list can be accessed with the command \cmdlink{nget}, using index notation as described on pg. \pageref{indexformat}. 
\begin{syntax}
\command{nget} \$ndlist \$i ...
\end{syntax}
\begin{args}
\$ndlist & ND list value. \\
\$i ... & Index arguments, using index notation (see pg. \pageref{indexformat}). The number of index arguments determines the interpreted dimensions.
\end{args}


\begin{example}{ND list access}
\begin{lstlisting}
set A {{1 2 3} {4 5 6} {7 8 9}}
puts [nget $A 0 :]
puts [nget $A 0* :]; # can "flatten" row
puts [nget $A 0:1 1]
puts [nget $A end:0 end:0]; # can have reverse ranges
puts [nget $A {0 0 0} 1*]; # can repeat indices
\end{lstlisting}
\tcblower
\begin{lstlisting}
{1 2 3}
1 2 3
2 5
{9 8 7} {6 5 4} {3 2 1}
2 2 2
\end{lstlisting}
\end{example}

\clearpage
\subsection{ND List Modification}
A ND list can be modified by reference with \cmdlink{nset}, and by value  with \cmdlink{nreplace}, using the same index argument syntax as \cmdlink{nget}. 
If the blank string is used as a replacement value, it will remove values from the ND lists, as long as it is only removing along one dimension. 
Otherwise, the replacement ND list must agree in dimension to the to the index argument dimensions, or be unity. 
For example, you can replace a 4x3 portion of a matrix with 4x3, 4x1, 1x3, or 1x1 matrices.
If modifying outside of the dimensions of the ND list, the ND list will be expanded to the new dimensions, like in the command \cmdlink{ndlist}.
\begin{syntax}
\command{nset} \$varName \$arg1 \$arg2 ... \$sublist
\end{syntax}
\begin{syntax}
\command{nreplace} \$ndlist \$arg1 \$arg2 ... \$sublist
\end{syntax}
\begin{args}
\$varName & Name of ND list to modify. \\
\$ndlist & ND list to modify. \\
\$arg1 \$arg2 ... & Index arguments, using index notation (see pg. \pageref{indexformat}). The number of index arguments determines the interpreted dimensions. \\
\$sublist & Replacement list, or blank to delete values.
\end{args}
\begin{example}{Swapping rows in a matrix}
\begin{lstlisting}
# ND List Value Modification
set a {{1 2} {3 4} {5 6}}
puts [nreplace $a : 1 ""]; # Delete a column (modify in-place)
nset a {1 0} : [nget $a {0 1} :]; # Swap rows and columns (modify by reference)
puts $a
\end{lstlisting}
\tcblower
\begin{lstlisting}
1 3 5
{3 4} {1 2} {5 6}
\end{lstlisting}
\end{example}


\clearpage
\section{ND Mapping}
The command \cmdlink{nmap} is a general purpose mapping function for N-dimensional lists in Tcl. 
If multiple ND lists are provided for iteration, they must agree in dimension or be unity, like in \cmdlink{nset}. 
Returns an ND list in similar fashion to the Tcl \textit{lmap} command. 
Additionally, elements can be skipped with \textit{continue}, and the entire loop can be exited with \textit{break}.
\begin{syntax}
\command{nmap} \$nd \$varName \$ndlist <\$varName \$ndlist ...> \$body
\end{syntax}
\begin{args}
\$nd & Dimensionality of ND list (e.g. 2D for a matrix).  \\
\$varName & Variable name to iterate with. \\
\$ndlist & ND list to iterate over. \\
\$body & Tcl script to evaluate at every loop iteration. 
\end{args}
\subsection{Map Index Access}
The iteration indices of \cmdlink{nmap} are accessed with the commands \cmdlink{i}, \cmdlink{j}, \& \cmdlink{k}. 
The commands \cmdlink{j} and \cmdlink{k} are simply shorthand for \cmdlink{i} with dimensions 1 and 2.
\begin{syntax}
\command{i} <\$axis>
\end{syntax}
\begin{syntax}
\command{j}
\end{syntax}
\begin{syntax}
\command{k}
\end{syntax}
\begin{args}	
\$axis & Dimension to access mapping index at. Default 0.
\end{args}
\begin{example}{ND list mapping}
\begin{lstlisting}
set testmat {{1 2 3} {4 5 6} {7 8 9}}
# Checkerboard sign pattern
puts [nmap 2D x $testmat {expr {
    $x*([i]%2 + [j]%2 == 1?-1:1)
}}]
# Simple formatting
puts [nmap 2D x $testmat {format %.2f $x}]
\end{lstlisting}
\tcblower
\begin{lstlisting}
{1 -2 3} {-4 5 -6} {7 -8 9}
{1.00 2.00 3.00} {4.00 5.00 6.00} {7.00 8.00 9.00}
\end{lstlisting}
\end{example}

\clearpage
\subsection{Basic Math Operations}
Basic math operators can be mapped over an ND list with the command \cmdlink{nop}. 
This method is much faster than \cmdlink{nmap} for simple math operations.

\begin{syntax}
\command{nop} \$nd \$ndlist \$op \$arg...
\end{syntax}
\begin{args}
\$nd & Dimensionality of ND list (e.g. 2D for a matrix).  \\
\$ndlist & ND list to perform element-wise operation over. \\
\$op & Math operator (using tcl::mathop namespace). \\
\$arg... & Operator arguments (see tcl::mathop documentation).
\end{args}

Additionally, this command is built-in to ND list objects with the ``\texttt{.=}'' operator, as shown below. 
\begin{example}{Element-wise operations}
\begin{lstlisting}
# Using ND list values
puts [nop 1D {1 2 3} -]
puts [nop 1D {1 2 3} + 1]
# Using ND list objects
matrix x {{1 2 3} {4 5 6}}
[$x .= {>= 3}] print
\end{lstlisting}
\tcblower
\begin{lstlisting}
-1 -2 -3
2 3 4
{0 0 1} {1 1 1}
\end{lstlisting}
\end{example}

\clearpage
\section{ND List Objects}
This package provides the ``ndlist'' type class, using the type system provided by the ``vutil'' package. 
The command \cmdlink{tensor} creates a new ND list object variable, which uses the class \texttt{::ndlist::ndobj}.
The commands \cmdlink{matrix}, \cmdlink{vector}, and \cmdlink{scalar} are shorthand for 2D, 1D, and 0D ND list objects.
\begin{syntax}
\command{tensor} \$refName \$nd <\$value>
\end{syntax}
\begin{syntax}
\command{matrix} \$refName <\$value>
\end{syntax}
\begin{syntax}
\command{vector} \$refName <\$value>
\end{syntax}
\begin{syntax}
\command{scalar} \$refName <\$value>
\end{syntax}
\begin{args}
\$refName & Name of reference variable to tie to object. \\
\$nd & Dimensionality of ND list (e.g. 2D for a matrix).  \\
\$value & ND list value.
\end{args}
\begin{example}{Create ND list object}
\begin{lstlisting}
matrix x {{1 2 3} {4 5 6} {7 8 9}}
puts [$x info]
\end{lstlisting}
\tcblower
\begin{lstlisting}
exists 1 ndims 2 shape {3 3} type ndlist value {{1 2 3} {4 5 6} {7 8 9}}
\end{lstlisting}
\end{example}

\clearpage
\subsection{Standard Methods}
Because the ND list objects are object variables, they have the same basic methods provided by the ``vutil'' package.
For more info on these methods, see the documentation for the ``vutil'' package.
\begin{syntax}
\method{ndlist}{-{}->} \$refName \\
\method{ndlist}{<-} \$object \\
\method{ndlist}{=} \$value \\
\method{ndlist}{.=} \$oper \\
\method{ndlist}{:=} \$expr \\
\method{ndlist}{::=} \$body \\
\method{ndlist}{info} <\$field> \\
\method{ndlist}{print} <-nonewline> <\$channelID> \\
\method{ndlist}{destroy}
\end{syntax}
\begin{args}
\$refName & Reference name to copy to. \\
\$object & ND list object. \\
\$value & ND list value, passed through the \cmdlink{ndlist} command. \\
\$oper & Math operator and arguments, evaluated with \cmdlink{nop}. \\
\$expr & Math expression, evaluated with \cmdlink{nexpr} \\
\$body & Tcl script to evaluate with \cmdlink{neval} \\
\$field & Field to query (fields ``shape'' and ``ndims'' added). \\
\$channelID & Open channel to print to. 
\end{args}

\begin{example}{ND list object methods}
\begin{lstlisting}
matrix X {{1 2} {3 4}}
$X ::= {format %0.2f $@.}; # Format values
$X print
$X --> Y; # Copy object
$Y .= {+ 1}; # Perform math operation
$Y print
\end{lstlisting}
\tcblower
\begin{lstlisting}
{1.00 2.00} {3.00 4.00}
{2.0 3.0} {4.0 5.0}
\end{lstlisting}
\end{example}

\clearpage
\subsection{Index Method}
The method ``\methodlink[0]{ndlist}{@}'' allows you to access or modify portions of an ND list, using the same index notation used by \cmdlink{nget}, \cmdlink{nreplace}, and \cmdlink{nset}.

\begin{syntax}
\protect\hypertarget{\$ndlistObj @}{\$ndlistObj @}\index{ndlist methods!"@} \$i ... <\$op \$input>
\end{syntax}
\begin{args}
\$ndlistObj & ND list object. \\
\$i ... & Index arguments, using index notation (see pg. \pageref{indexformat}). The number of index arguments must match the dimensionality of the ND list object. \\
\$op \$input & Operator to perform on the sublist. Default simply returns value. \\
\quad -{}-> \$refName & \quad Create new object from range with reference name \texttt{\$refName}. \\
\quad <- \$object & \quad Assign value of \texttt{\$object} to range (must have same dimensionality). \\
\quad = \$value  & \quad Assign \texttt{\$value} to range after passing through the \cmdlink{ndlist} command. \\
\quad .= \$oper & \quad Modify range in place using \cmdlink{nop}. \\
\quad := \$expr & \quad Modify range in place using \cmdlink{nexpr}. \\
\quad ::= \$body & \quad Modify range in place using \cmdlink{neval}. 
\end{args}

\begin{example}{ND list object access/manipulation}
\begin{lstlisting}
# Access ND List Objects
matrix X {{1 2} {3 4}}
puts [$X @ : 1]; # get column value
$X @ 1* : --> Y; # create row vector (1D list)
$Y @ end .= {* 2}; # double last element of Y
puts [$Y info]
\end{lstlisting}
\tcblower
\begin{lstlisting}
2 4
exists 1 ndims 1 shape 2 type ndlist value {3 8}
\end{lstlisting}
\end{example}
\clearpage


\section{Object Reference Mapping}
Similar to the commands \textit{leval} and \textit{lexpr} in the ``vutil'' package, the commands \cmdlink{neval} and \cmdlink{nexpr} perform element-wise operations over ND list objects.
Both use the command \cmdlink{nmap}, so you have access to the index commands \cmdlink{i}, \cmdlink{j}, and \cmdlink{k}.
Additionally, these are built into the ``\texttt{:=}'' and ``\texttt{::=}'' ND list operators.
\begin{syntax}
\command{neval} \$body <\$nd \$ndlist> <"-{}->" \$refName>
\end{syntax}
\begin{syntax}
\command{nexpr} \$expr <\$nd \$ndlist> <"-{}->" \$refName>
\end{syntax}
\begin{args}
\$body & Tcl script with list object references. \\
\$expr & Tcl expression with list object references. \\
\$nd & Dimensionality of ND list (e.g. 2D for a matrix).  \\
\$ndlist & ND list to iterate over, with \texttt{\$@.} reference. \\
\$refName & Reference variable to tie resulting ND list to. Default blank returns value.
\end{args}

\begin{example}{Element-wise expressions}
\begin{lstlisting}
matrix x {{1 2} {3 4} {5 6}}
matrix y 5.0
puts [nexpr {$@x + $@y}]
\end{lstlisting}
\tcblower
\begin{lstlisting}
{6.0 7.0} {8.0 9.0} {10.0 11.0}
\end{lstlisting}
\end{example}

\begin{example}{Self-operation, using index access commands}
\begin{lstlisting}
matrix x [nrepeat {2 3} 1]
[$x := {$@. * [i]}] print
\end{lstlisting}
\tcblower
\begin{lstlisting}
{0 0 0} {1 1 1}
\end{lstlisting}
\end{example}
\clearpage
\section{Dimensionality, Shape and Size}
ND list objects store the dimensionality of the list, which can be accessed or changed with the method \methodlink[0]{ndlist}{ndims}.
\begin{syntax}
\method{ndlist}{ndims} <\$nd> 
\end{syntax}
\begin{args}
\$nd & Dimensionality of ND list (e.g. 2D for a matrix).  
\end{args}
The shape of an ND list can be accessed with the command \cmdlink{nshape}, and the total size can be accessed with the command \cmdlink{nsize}.
For ND list objects, the methods \methodlink[0]{ndlist}{shape} and \methodlink[0]{ndlist}{size} access the shape and size of the value stored in the object.
\begin{syntax}
\command{nshape} \$nd \$ndlist <\$axis> \\
\method{ndlist}{shape} <\$axis>
\end{syntax}
\begin{syntax}
\command{nsize} \$nd \$ndlist \\
\method{ndlist}{size}
\end{syntax}
\begin{args}
\$nd & Dimensionality of ND list (e.g. 2D for a matrix).  \\
\$ndlist & ND list to get dimensions of. \\
\$axis & Axis to get dimension along. Blank for all. \\
\end{args}

\begin{example}{Shape and size}
\begin{lstlisting}
set x {{1 2} {3 4} {5 6}}
puts [nshape 2D $x]
puts [nsize 2D $x]
# Convert scalar ND list object to matrix
scalar x 5.0
$x ndims 2
puts [$x shape]
\end{lstlisting}
\tcblower
\begin{lstlisting}
3 2
6
1 1
\end{lstlisting}
\end{example}

\clearpage
\section{Flattening and Reshaping}
The command \cmdlink{nflatten} flattens an ND list, and the command \cmdlink{nreshape} flattens and reshapes an ND list to a compatible shape.
For ND list objects, the methods \methodlink[0]{ndlist}{flatten} and \methodlink[0]{ndlist}{reshape} flatten and reshape the value stored in the object.
\begin{syntax}
\command{nflatten} \$nd \$ndlist \\
\method{ndlist}{flatten}
\end{syntax}
\begin{args}
\$nd & Dimensionality of ND list (e.g. 2D for a matrix).  \\
\$ndlist & ND list to flatten. 
\end{args}
\begin{syntax}
\command{nreshape} \$nd \$ndlist \$shape \\
\method{ndlist}{reshape} \$shape
\end{syntax}
\begin{args}
\$nd & Dimensionality of ND list (e.g. 2D for a matrix).  \\
\$ndlist & ND list to flatten and reshape. \\
\$shape & New shape (list of dimensions).
\end{args}

\begin{example}{Flatten and reshape ND lists}
\begin{lstlisting}
vector x [nflatten 2D {{1 2 3 4} {5 6 7 8}}]
[$x reshape {2 2 2}] print
\end{lstlisting}
\tcblower
\begin{lstlisting}
{{1 2} {3 4}} {{5 6} {7 8}}
\end{lstlisting}
\end{example}

\clearpage

\section{Transpose or Swap Axes}
The command \cmdlink{ntranspose} or the method \methodlink[0]{ndlist}{transpose} swaps axes of an ndlist. 
By default, it just transposes the matrix representation of the data, swapping rows and columns.
For ND list objects, the method \methodlink[0]{ndlist}{transpose} transposes the value stored in the object.
\begin{syntax}
\command{ntranspose} \$nd \$ndlist <\$axis1 \$axis2> \\
\method{ndlist}{transpose} <\$axis1 \$axis2>
\end{syntax}
\begin{args}
\$nd & Dimensionality of ND list (e.g. 2D for a matrix).  \\
\$ndlist & ND list to manipulate. \\
\$axis1 & Axis to swap with axis 2 (default 0) \\
\$axis2 & Axis to swap with axis 1 (default 1)
\end{args}
\begin{example}{Transposing a matrix}
\begin{lstlisting}
puts [ntranspose 2D {{1 2} {3 4}}]
\end{lstlisting}
\tcblower
\begin{lstlisting}
{1 3} {2 4}
\end{lstlisting}
\end{example}
\begin{example}{Swapping axes of a tensor}
\begin{lstlisting}
tensor x 3D {{{1 2} {3 4}} {{5 6} {7 8}}}
[$x transpose 0 2] print
\end{lstlisting}
\tcblower
\begin{lstlisting}
{{1 5} {3 7}} {{2 6} {4 8}}
\end{lstlisting}
\end{example}
\clearpage
\section{Combine ND Lists}
The command \cmdlink{ninsert} allows you to insert a sublist into an ND list at a specified index and axis. 
Sublist must agree in dimension at all other axes.
For ND list objects, the method \methodlink[0]{ndlist}{insert} inserts a sublist into the value stored in the object.
\begin{syntax}
\command{ninsert} \$nd \$ndlist \$index \$sublist <\$axis> \\
\method{ndlist}{insert} \$index \$sublist <\$axis>
\end{syntax}
\begin{args}
\$nd & Dimensionality of ND list (e.g. 2D for a matrix).  \\
\$ndlist & ND list to manipulate. \\
\$index & Index to insert at. \\
\$axis & Axis to insert at (default 0).
\end{args}

\begin{example}{Inserting rows and columns in a matrix}
\begin{lstlisting}
# Insert row
puts [ninsert 2D {{1 2 3} {4 5 6} {7 8 9}} 0 {{A B C}}]
# Insert column
matrix x {1 2 3}
$x insert end {4 5 6} 1
$x print
\end{lstlisting}
\tcblower
\begin{lstlisting}
{A B C} {1 2 3} {4 5 6} {7 8 9}
{1 4} {2 5} {3 6}
\end{lstlisting}
\end{example}

\begin{example}{Stack tensors}
\begin{lstlisting}
set x [nreshape 1D {1 2 3 4 5 6 7 8 9} {3 3 1}]
set y [nreshape 1D {A B C D E F G H I} {3 3 1}]
puts [ninsert 3D $x end $y 2]
\end{lstlisting}
\tcblower
\begin{lstlisting}
{{1 A} {2 B} {3 C}} {{4 D} {5 E} {6 F}} {{7 G} {8 H} {9 I}}
\end{lstlisting}
\end{example}

\clearpage

\section{Filling Blanks}
Modifying an ND list outside of its dimensions automatically expands the list with blanks to match the new shape. 
The command \cmdlink{nfill} replaces blank values with a specified filler value.
For ND list objects, the method \methodlink[0]{ndlist}{fill} fills blanks in the value stored in the object.
\begin{syntax}
\command{nfill} \$nd \$ndlist \$filler \\
\method{ndlist}{fill} \$filler
\end{syntax}
\begin{args}
\$nd & Dimensionality of ND list (e.g. 2D for a matrix).  \\
\$ndlist & ND list to iterate over. \\
\$filler & Filler to replace blanks with.
\end{args}

\begin{example}{Creating an identity matrix}
\begin{lstlisting}
set I ""
for {set i 0} {$i < 3} {incr i} {
    nset I $i $i 1
}
set I [nfill 2D $I 0]
puts $I
\end{lstlisting}
\tcblower
\begin{lstlisting}
{1 0 0} {0 1 0} {0 0 1}
\end{lstlisting}
\end{example}
\clearpage
\section{ND List Indexing}\label{indexformat}
Index input for all ND list access and modification functions (\cmdlink{nget}, \cmdlink{nreplace}, \cmdlink{nset} and the ND list object method ``\methodlink[0]{ndlist}{@}'') gets passed through the ND list index parser \cmdlink{::ndlist::ParseIndex}.
\begin{syntax}
\command{::ndlist::ParseIndex} \$input \$n
\end{syntax}
\begin{args}
\$input & Index input. Options are shown below: \\
\quad : & All indices \\
\quad \$start:\$stop & Range of indices (e.g. 0:4 or 1:end-2).\\
\quad \$start:\$step:\$stop & Stepped range of indices (e.g. 0:2:-2 or 2:3:end), using \cmdlink{range}. \\
\quad \$iList & List of indices (e.g. \{0 end-1 5\} or 3). \\
\quad \$i* & Single index with asterisk, ``flattens'' the ndlist (e.g. 0* or end-3*). \\
\$n & Number of elements in list.
\end{args}
Additionally, index range arguments \texttt{\$start} and \texttt{\$stop}, all indices in \texttt{\$iList}, and single indices \texttt{\$i} get passed through the \cmdlink{::ndlist::Index2Integer} command, which converts \texttt{end}$\pm$\textit{integer}, \textit{integer}$\pm$\textit{integer} and negative wrap-around indexing (where -1 is equivalent to ``end'') into normal integer indices.
\begin{syntax}
\command{::ndlist::Index2Integer} \$index \$n
\end{syntax}
\begin{args}
\$index & Single index. \\
\$n & Number of elements in list.
\end{args}


\clearpage
\subsection{Range Generator}
The index range notation ``\texttt{\$start:\$step:\$stop}'' uses the command \cmdlink{range}, which simply generates a list of integer values. 
\begin{syntax}
\command{range} \$n \\
range \$start \$stop <\$step>
\end{syntax}
\begin{args}
\$n & Number of indices, starting at 0 (e.g. 3 returns 0 1 2). \\
\$start & Starting value. \\
\$stop & Stop value. \\
\$step & Step size. Default 1 or -1, depending on direction of start to stop.
\end{args}
\begin{example}{Integer range generator}
\begin{lstlisting}
puts [range 3]
puts [range 0 2]
puts [range 10 3 -2]
# Alternative for-loop
foreach i [range 5] {
    puts $i
}
\end{lstlisting}
\tcblower
\begin{lstlisting}
0 1 2
0 1 2
10 8 6 4
0
1
2
3
4
\end{lstlisting}
\end{example}
\clearpage
{\normalsize\printindex}
\end{document}
