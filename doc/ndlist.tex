\documentclass{article}

% Input packages & formatting
% Packages

% Math packages
\usepackage{amsmath} % Extended math functions
\usepackage{amssymb} % Extended math symbols (loads in amsfonts)
\usepackage{bm} % Bold math symbols
\usepackage{mathtools}

% Figure packages
\usepackage{caption} % Caption formatting for university standard
\usepackage{graphicx} % includegraphics command
\usepackage{subcaption} % Subfigures
\usepackage[section]{placeins} % Place floats in section
\usepackage{wrapfig}

% Table packages
\usepackage{booktabs} % Better tables
\usepackage{bigstrut} % Merged table cells
\usepackage{longtable} % Tables which overflow into next page
\usepackage{array}
\usepackage{colortbl} % Color table cells
\usepackage{makecell}
\usepackage{multirow}

% Fonts
\usepackage{lmodern} % Use latin modern rather than computer modern. Better for font encoding.
\usepackage[T1]{fontenc} % Allow text to be searchable in output

% Other packages
\usepackage{appendix} % Appendix environment
\usepackage{nextpage} % Cleartooddpage command
%\usepackage[square,comma,sort,numbers]{natbib} % Reference formatting
\usepackage{setspace} % Line spacing
\usepackage{listings} % Display code with syntax highlighting
\usepackage{upquote} % Vertical quotes in verbatim
\usepackage{xcolor} % Colors
\usepackage{titlesec} % Header spacing
\usepackage{xparse} % for tcolorbox
\usepackage[listings]{tcolorbox} % Colored boxes for highlighting syntax
\tcbuselibrary{breakable}
\tcbuselibrary{skins}
\usepackage{enumitem} % better enumerate/itemize options
\usepackage{fancyhdr}
\usepackage{multicol}
\usepackage{ifthen}
\usepackage{xstring}

% Table of contents
\usepackage{imakeidx} % Index page
\usepackage{tocloft} % Control of table of contents
\usepackage[nottoc]{tocbibind} % Adds bibliography, table of tables, table of figures, to table of contents
\usepackage[bookmarks,linktocpage,hidelinks]{hyperref} % Hyperlinks for sections, figures, etc.

% Formatting
% Page format
\setlength{\oddsidemargin}{0.00in}  % Left side margin for odd numbered pages
\setlength{\evensidemargin}{0.00in} % Right side margin for even numbered pages
\setlength{\topmargin}{0.00in}      % Top margin
\setlength{\headheight}{0.20in}     % Header height
\setlength{\headsep}{0.20in}        % Separation between header and main text
\setlength{\topskip}{0.00in}        % Top skip
\setlength{\textwidth}{6.50in}      % Width of the text
\setlength{\textheight}{8.50in}     % Height of the text
\setlength{\footskip}{0.50in}       % Foot skip
\setlength{\parindent}{0.00in}      % First line indentation
\setlength{\parskip}{6pt}        % Space between two paragraphs

% Captions (figures, tables, etc.)
\setlength{\floatsep}{\parskip}          % Space left between floats.
\setlength{\textfloatsep}{\floatsep}   % Space between last top float
% or first bottom float and the text
\setlength{\intextsep}{\floatsep}      % Space left on top and bottom
% of an in-text float
\setlength{\abovecaptionskip}{0.1in plus 0.25in}  % Space above caption
\setlength{\belowcaptionskip}{0.1in plus 0.25in}  % Space below caption
\setlength{\captionmargin}{0.50in}     % Left/Right margin for caption
\setlength{\abovedisplayskip}{0.00in plus 0.25in} % Space before Math stuff
\setlength{\belowdisplayskip}{0.00in plus 0.25in} % Space after Math stuff
\setlength{\arraycolsep}{0.10in}       % Gap between columns of an array
\setlength{\jot}{0.10in}                % Gap between multiline equations
\setlength{\itemsep}{0.10in}           % Space between successive items

% Counters (no section numbering)
\setcounter{tocdepth}{3}
\setcounter{secnumdepth}{0}

% Spacing
\setstretch{1.5}

\titlespacing*{\section}{0cm}{6pt}{6pt}[0cm]
\titlespacing*{\subsection}{0cm}{6pt}{6pt}[0cm]
\titlespacing*{\subsubsection}{0cm}{6pt}{6pt}[0cm]

\titleformat{\section}
{\sffamily\huge}{}{0pt}{\titlerule\vspace{-0.2cm}}
\titleformat{\subsection}
{\sffamily\itshape\Large}{}{0pt}{}

% Macro for syntax
\newtcolorbox{syntax}{
    size=small,
    sharp corners,
    colframe=black,
    colback=yellow,
    fontupper=\bfseries\ttfamily
}

% Macro for argument table
\newenvironment{args}{
    \begin{tabular}{>{\bfseries\ttfamily}p{0.25\linewidth} p{0.69\linewidth}}
    }{
    \end{tabular}\par
    \vspace{0.5\baselineskip}
}

% Note: Requires packages "listing", "xcolor", and "textcomp"
\lstdefinelanguage{verbatim}{
    basicstyle=\ttfamily\small,
    xleftmargin=9pt,
    xrightmargin=9pt,
    columns=fullflexible,
    keepspaces=true,
    comment=[l]{\#},
    breaklines=true
}

\lstdefinestyle{verbatim}{
    commentstyle=\color{gray},
}

% Example code
\AtBeginDocument{
\newtcolorbox[blend into=listings]{example}[2][]{
    colback=blue!3!white,
    colframe=black,
    colbacktitle=blue!15!white,
    coltitle=black,
    sharp corners,
    enhanced,
    breakable,
    size=small,
    before upper={
        \setstretch{1.0}\lstset{language=verbatim,style=verbatim}\vspace{3pt}\textsf{\textit{Code:}}
    },
    subtitle style={
        colback=blue!20!white,
        fonttitle=\sffamily
    },
    before lower={
        \setstretch{1.0}\lstset{language=verbatim,style=verbatim}\vspace{3pt}\textsf{\textit{Output:}}
    },
    fonttitle=\sffamily,
    title={#2},
    #1
}
}

% Links to sub and subsub commands - optional boolean argument, default true. if false, only displays subcmd.

% Commands (and command ensembles)
\newcommand{\command}[1]{\protect\hypertarget{#1}{#1}\index{#1}}
\newcommand{\subcommand}[2]{\protect\hypertarget{#1 #2}{#1 #2}\index{#1!#2}}
\newcommand{\cmdlink}[1]{\protect\hyperlink{#1}{\textit{#1}}}
\newcommand{\subcmdlink}[3][1]{\protect\hyperlink{#2 #3}{\ifnum#1=1\relax\textit{#2 #3}\else\textit{#3}\fi}}

% Methods (first arg is class)
\newcommand{\method}[2]{\protect\hypertarget{$#1Obj #2}{\$#1Obj #2}\index{#1 methods!#2}}
\newcommand{\methodlink}[3][1]{\protect\hyperlink{$#2Obj #3}{\ifnum#1=1\relax\textit{\$#2Obj #3}\else\textit{#3}\fi}}

% Macros for figure/table names
\newcommand{\fig}{\figurename\ }
\newcommand{\figs}{\figurename s }
\newcommand{\tbl}{\tablename\ }
\newcommand{\tbls}{\tablename s }
\newcommand{\eq}{Eq. }
\newcommand{\eqs}{Eqs. }
\renewcommand{\lstlistingname}{Example}% Listing -> Example
\renewcommand{\lstlistlistingname}{List of \lstlistingname s}% List of Listings -> List of Examples
\newcommand{\ex}{Example }
\newcommand{\exs}{Examples }
\newcommand{\var}[1]{\texttt{\textbf{\$#1}}}

% Header/footer
\renewcommand{\headrulewidth}{0pt}

% Changes to hyperlinks (URLs)
\renewcommand\UrlFont{\color{blue}\rmfamily}

% New column type 
% https://tex.stackexchange.com/questions/75717/how-can-i-mix-itemize-and-tabular-environments
\newcolumntype{L}{>{\labelitemi~~}l<{}}
\newcommand{\version}{0.3.1}

\renewcommand{\cleartooddpage}[1][]{\ignorespaces} % single side
\newcommand{\caret}{$^\wedge$}

\title{\Huge{N-Dimensional Lists (ndlist)}\\\large Version \version}
\author{Alex Baker\\\small\url{https://github.com/ambaker1/ndlist}}
\date{\small\today}
\begin{abstract}
\begin{center}
The ``ndlist'' module provides tools for vector, matrix, and tensor manipulation and processing, where vectors are represented by Tcl lists, and matrices are represented by nested Tcl lists, and higher dimension lists represented by additional levels of nesting. 
Additionally, this package provides the ``ndlist'' object variable type.
\end{center}
\end{abstract}

\begin{document}
\section{N-Dimensional Lists}
A ND list is defined as a list of equal length (N-1)D lists, which are defined as equal length (N-2)D lists, and so on until (N-N)D lists, which are scalars of arbitrary size.
For example, a matrix is a 2D list, or a list of equal length row vectors (1D), which contain arbitrary scalar values, as shown below:
\begin{equation*}\label{eq:matrix_AB}
A=\begin{bmatrix}
2 & 5 & 1 & 3 \\
4 & 1 & 7 & 9 \\
6 & 8 & 3 & 2 \\
7 & 8 & 1 & 4
\end{bmatrix},\quad
B=\begin{bmatrix}
9 \\ 3 \\ 0 \\ -3
\end{bmatrix},\quad
C = \begin{bmatrix}
3 & 7 & -5 & -2
\end{bmatrix}
\end{equation*}
\begin{example}[label=ex:matrix_AB]{Defining matrices in Tcl}
\begin{lstlisting}
set A {{2 5 1 3} {4 1 7 9} {6 8 3 2} {7 8 1 4}}
set B {9 3 0 -3}
set C {{3 7 -5 -2}}
\end{lstlisting}
\end{example}

This definition is flexible, and allows for different interpretations of the same data. For example, the list ``1 2 3'' can be interpreted as a scalar with value ``1 2 3'', a vector with values ``1'', ``2'', and ``3'', or a matrix with row vectors ``1'', ``2'', and ``3''.

\subsection{Transposing}
The command \cmdlink{nswap} swaps axes of an ndlist. By default, it just transposes the matrix representation of the data, swapping rows and columns.
\begin{syntax}
\command{nswap} \$ndlist <\$axis1 \$axis2>
\end{syntax}
\begin{args}
\$ndlist & ND list to manipulate. \\
\&axis1 & Axis to swap with axis 2 (default 0) \\
\&axis2 & Axis to swap with axis 1 (default 1)
\end{args}
Returns an mxn matrix.
\begin{example}{Transposing a matrix}
\begin{lstlisting}
puts [nswap {{1 2} {3 4}}]
\end{lstlisting}
\tcblower
\begin{lstlisting}
{1 3} {2 4}
\end{lstlisting}
\end{example}


# Reshape implies reshaping

\section{}
\begin{syntax}
\command{ndeserialize} \$list \$n1 \$n2 ...
\end{syntax}
\begin{args}
\$vector & Vector to reshape \\
\$n & Number of rows in new matrix \\
\$m & Number of columns in new matrix
\end{args}
\begin{example}{Flattening and reshaping matrices}
\begin{lstlisting}
puts [flatten {{1 2 3} {4 5 6} {7 8 9}}]
puts [reshape {1 2 3 4 5 6} 3 2]
\end{lstlisting}
\tcblower
\begin{lstlisting}
1 2 3 4 5 6 7 8 9
{1 2} {3 4} {5 6}
\end{lstlisting}
\end{example}

\clearpage
\section{Matrices (2D)}
Matrices are represented in Tcl by nested lists, where each sublist is a row vector.

\clearpage
\subsection{Flattening and Reshaping}
The command \cmdlink{flatten} flattens a matrix to a 1D vector, while the command \cmdlink{reshape} reshapes a 1D vector into a compatible 2D matrix. 
\begin{syntax}
\command{flatten} \$matrix
\end{syntax}
\begin{args}
\$matrix & Matrix to flatten
\end{args}
\begin{syntax}
\command{reshape} \$vector \$n \$m
\end{syntax}
\begin{args}
\$vector & Vector to reshape \\
\$n & Number of rows in new matrix \\
\$m & Number of columns in new matrix
\end{args}
\begin{example}{Flattening and reshaping matrices}
\begin{lstlisting}
puts [flatten {{1 2 3} {4 5 6} {7 8 9}}]
puts [reshape {1 2 3 4 5 6} 3 2]
\end{lstlisting}
\tcblower
\begin{lstlisting}
1 2 3 4 5 6 7 8 9
{1 2} {3 4} {5 6}
\end{lstlisting}
\end{example}
\clearpage
\subsection{Stacking and Augmenting Matrices}
The commands \cmdlink{stack} and \cmdlink{augment} can be used to combined matrices, row or column-wise.
Matrices can be combined row-wise or column-wise with the commands \cmdlink{stack} \& \cmdlink{augment}. 
\begin{syntax}
\command{stack} \$mat1 \$mat2 ...
\end{syntax}
\begin{syntax}
\command{augment} \$mat1 \$mat2 ...
\end{syntax}
\begin{args}
\$mat1 \$mat2 ... & Arbitrary number of matrices to stack/augment (number of columns/rows must match)
\end{args}
\begin{example}{Combining matrices}
\begin{lstlisting}
puts [stack {{1 2}} {{3 4}}]
puts [augment {1 2} {3 4}]
\end{lstlisting}
\tcblower
\begin{lstlisting}
{1 2} {3 4}
{1 3} {2 4}
\end{lstlisting}
\end{example}
\clearpage
\subsection{Matrix Multiplication}
The command \cmdlink{matmul} performs matrix multiplication for two matrices. Adapted from \textit{matmul} from the Tcllib math::linearalgebra package, with a few additions. First of all, scalars are considered to be valid matrices, and if more than two matrices are provided, the order of multiplication will be optimized, as described in ``Introduction to Algorithms'' \cite{cormen_introduction_2001}.
\begin{syntax}
\command{matmul} \$A \$B <\$C \$D ...>
\end{syntax}
\begin{args}
\$A & Left matrix, nxq. \\
\$B & Right matrix, qxm. \\
\$C \$D ... & Additional matrices to multiply (optional). 
\end{args}
Returns an nxm matrix (or the corresponding dimensions from additional matrices)
\begin{example}{Multiplying a matrix}
\begin{lstlisting}
puts [matmul {{2 5 1 3} {4 1 7 9} {6 8 3 2} {7 8 1 4}} {9 3 0 -3}]
\end{lstlisting}
\tcblower
\begin{lstlisting}
24.0 12.0 72.0 75.0
\end{lstlisting}
\end{example}
\clearpage
\subsection{Cartesian Product}
The command \cmdlink{cartprod} computes the Cartesian product of an arbitrary number of vectors, returning a matrix where the columns correspond to the input vectors and the rows correspond to all the combinations of the vector elements.

\begin{syntax}
\command{cartprod} \$list1 \$list2 ...
\end{syntax}
\begin{args}
\$list1 \$list2 ... & Lists, or vectors, to take Cartesian product of.
\end{args}

Similarly, the command \cmdlink{cartgrid} returns all combinations of input parameters and lists.
\begin{syntax}
\command{cartgrid} \$dict \\
cartgrid \$keys \$list <\$keys \$list ...>
\end{syntax}
\begin{args}
\$dict & Dictionary of keys and lists. \\
\$keys & List of parameter names. \\
\$list & Parameter value list.
\end{args}

\begin{example}[label=ex:cartgrid]{Nested parameter study without nested loops}
\begin{lstlisting}
dict set params a {1 2}
dict set params b {3 4}
dict set params c {5 6}
foreach line [cartgrid $params] {
    puts $line
}
\end{lstlisting}
\tcblower
\begin{lstlisting}
a 1 b 3 c 5
a 1 b 3 c 6
a 1 b 4 c 5
a 1 b 4 c 6
a 2 b 3 c 5
a 2 b 3 c 6
a 2 b 4 c 5
a 2 b 4 c 6
\end{lstlisting}
\end{example}
\clearpage
\section{N-Dimensional Lists}


\subsection{Creation}
ND lists can be initialized with \cmdlink{nrepeat}. This is similar to \textit{lrepeat}, except that it generates nested lists. Aliases for matrices (2D) and vectors (1D) are provided with the commands \cmdlink{mrepeat} and \cmdlink{vrepeat}.
\begin{syntax}
\command{nrepeat} \$n \$m ... \$value
\end{syntax}
\begin{syntax}
\command{mrepeat} \$n \$m \$value
\end{syntax}
\begin{syntax}
\command{vrepeat} \$n \$value
\end{syntax}
\begin{args}
\$n \$m ... & Shape of ND list. \\
\$value & Value to repeat.
\end{args}
\begin{example}{Create nested ND list with one value}
\begin{lstlisting}
nrepeat 2 2 2 0
\end{lstlisting}
\tcblower
\begin{lstlisting}
{{0 0} {0 0}} {{0 0} {0 0}}
\end{lstlisting}
\end{example}
\clearpage
\subsection{Shape}
The shape (dimensions) of an ND list can be queried with \cmdlink{nshape}. 
Simply takes the list lengths along index zero, assuming that all other sublists are the same length.
Aliases for matrices (2D) and vectors (1D) are provided with the commands \cmdlink{mshape} and \cmdlink{vshape}.
\begin{syntax}
\command{nshape} \$ndtype \$ndlist <\$dim>
\end{syntax}
\begin{syntax}
\command{mshape} \$matrix <\$dim>
\end{syntax}
\begin{syntax}
\command{vshape} \$vector
\end{syntax}
\begin{args}
\$ndtype & Type of ND list. (e.g. 2D for matrix). \\
\$ndlist & ND list to get shape for. \\
\$dim & Dimension to get (e.g. 0 gets number of rows in a matrix). By default returns list of all dimensions. 
\end{args}
\clearpage
\subsection{Access}
Portions of an ND list can be accessed with the command \cmdlink{nget}.
Aliases for matrices (2D) and vectors (1D) are provided with the commands \cmdlink{mget} and \cmdlink{vget}, and aliases for accessing matrix rows and columns (using \$i* indexing), are provided with the commands \cmdlink{rget} and \cmdlink{cget}.
\begin{syntax}
\command{nget} \$ndlist \$arg1 \$arg2 ...
\end{syntax}
\begin{syntax}
\command{mget} \$matrix \$i \$j
\end{syntax}
\begin{syntax}
\command{rget} \$matrix \$i
\end{syntax}
\begin{syntax}
\command{cget} \$matrix \$j
\end{syntax}
\begin{syntax}
\command{vget} \$vector \$i
\end{syntax}
\begin{args}
\$ndlist & ND list to access \\
\$arg1 \$arg2 ... & Index arguments, using index notation explained on page \pageref{indexformat}. The number of index arguments determines the interpreted dimensions.
\end{args}

\begin{example}{Accessing matrix values, rows, and columns}
\begin{lstlisting}
set A {{1 2 3} {4 5 6} {7 8 9}}
puts [mget $A 0 0]
puts [rget $A 0]
puts [cget $A 0]
\end{lstlisting}
\tcblower
\begin{lstlisting}
1
1 2 3
1 4 7
\end{lstlisting}
\end{example}

\clearpage
\subsection{Index Notation}\label{indexformat}
Index input for all ND list access and modification functions gets passed through the ND list index parser \cmdlink{::ndlist::ParseIndex}.
\begin{syntax}
\command{::ndlist::ParseIndex} \$input \$n
\end{syntax}
\begin{args}
\$input & Index input. Options are shown below: \\
\quad : & All indices \\
\quad \$start:\$stop & Range of indices (e.g. 0:4 or 1:end-2).\\
\quad \$start:\$step:\$stop & Stepped range of indices (e.g. 0:2:-2 or 2:3:end). \\
\quad \$iList & List of indices (e.g. \{0 end-1 5\} or 3). \\
\quad \$i* & Single index with asterisk, ``flattens'' the ndlist (e.g. 0* or end-3*). \\
\$n & Number of elements in list.
\end{args}
Additionally, index range arguments \texttt{\$start} and \texttt{\$stop}, all indices in \texttt{\$iList}, and single indices \texttt{\$i} get passed through the \cmdlink{::ndlist::Index2Integer} command, which converts \texttt{end}$\pm$\textit{integer}, \textit{integer}$\pm$\textit{integer} and negative wrap-around indexing (where -1 is equivalent to ``end'') into normal integer indices.
\begin{syntax}
::ndlist::Index2Integer \$index \$n
\end{syntax}
\begin{args}
\$index & Single index. \\
\$n & Number of elements in list.
\end{args}

\begin{example}{Index notation}
\begin{lstlisting}
set A {{1 2 3} {4 5 6} {7 8 9}}
puts [mget $A 0 :]
puts [mget $A 0* :]; # can "flatten" row
puts [mget $A 0:1 1]
puts [mget $A end:0 end:0]; # can have reverse ranges
puts [mget $A {0 0 0} 1*]; # can repeat indices
\end{lstlisting}
\tcblower
\begin{lstlisting}
{1 2 3}
1 2 3
2 5
{9 8 7} {6 5 4} {3 2 1}
2 2 2
\end{lstlisting}
\end{example}
\clearpage
\subsection{Modification by Reference}
A ND list can be modified by reference with \cmdlink{nset}, using the same index argument syntax as \cmdlink{nget}. 
If the blank string is used as a replacement value, it will remove values from the ND lists, as long as it is only removing along one dimension. 
Otherwise, the replacement ND list must agree in dimension to the to the index argument dimensions, or be unity. 
For example, you can replace a 4x3 portion of a matrix with 4x3, 4x1, 1x3, or 1x1 matrices.
Aliases for matrices (2D) and vectors (1D) are provided with the commands \cmdlink{mset} and \cmdlink{vset}, and aliases for modifying matrix rows and columns (using \$i* indexing), are provided with the commands \cmdlink{rset} and \cmdlink{cset}.
\begin{syntax}
\command{nset} \$varName \$arg1 \$arg2 ... \$sublist
\end{syntax}
\begin{syntax}
\command{mset} \$varName \$i \$j \$submat
\end{syntax}
\begin{syntax}
\command{rset} \$varName \$i \$subrow
\end{syntax}
\begin{syntax}
\command{cset} \$varName \$j \$subcol
\end{syntax}
\begin{syntax}
\command{vset} \$varName \$i \$subvec
\end{syntax}
\begin{args}
\$varName & Name of ND list to modify \\
\$arg1 \$arg2 ... & Index arguments, using index notation explained on page \pageref{indexformat}. The number of index arguments determines the interpreted dimensions. \\
\$sublist & Compatible ND list to replace at the specified indices, or blank to remove values.
\end{args}
\begin{example}{Swapping rows in a matrix}
\begin{lstlisting}
set a {{1 2} {3 4} {5 6}}
nset a {1 0} : [nget $a {0 1} :]
puts $a
\end{lstlisting}
\tcblower
\begin{lstlisting}
{3 4} {1 2} {5 6}
\end{lstlisting}
\end{example}
Note: if attempting to modify outside of the dimensions of the ND list, the ND list will be expanded and filled with the value in the variable \texttt{::ndlist::filler}. By default, the filler is 0, but this can easily be changed.
\clearpage
\subsection{Modification by Value}
In the same fashion as \cmdlink{nset}, an ND list can be modified by value with \cmdlink{nreplace}, returning a new ND list.
Aliases for matrices (2D) and vectors (1D) are provided with the commands \cmdlink{mreplace} and \cmdlink{vreplace}, and aliases for modifying matrix rows and columns (using \$i* indexing), are provided with the commands \cmdlink{rreplace} and \cmdlink{creplace}.
\begin{syntax}
\command{nreplace} \$ndlist \$arg1 \$arg2 ... \$sublist
\end{syntax}
\begin{syntax}
\command{mreplace} \$matrix \$i \$j \$submat
\end{syntax}
\begin{syntax}
\command{rreplace} \$matrix \$i \$subrow
\end{syntax}
\begin{syntax}
\command{creplace} \$matrix \$j \$subcol
\end{syntax}
\begin{syntax}
\command{vreplace} \$vector \$i \$subvec
\end{syntax}
\begin{args}
\$ndlist & ND list to modify. Returns new ND list. \\
\$arg1 \$arg2 ... & Index arguments, using index notation explained on page \pageref{indexformat}. The number of index arguments determines the interpreted dimensions. \\
\$sublist & Compatible ND list to replace at the specified indices, or blank to remove values.
\end{args}
\clearpage
\subsection{Functional Mapping}
A functional map can be applied over an ND list with \cmdlink{nmap}. 
Note that this differs significantly from the Tcl \textit{lmap} command.
Aliases for matrices (2D) and vectors (1D) are provided with the commands \cmdlink{mmap} and \cmdlink{vmap}.
Aliases for mapping over matrix rows and columns are provided with the commands \cmdlink{rmap} and \cmdlink{cmap}.
\begin{syntax}
\command{nmap} \$ndtype \$command \$ndlist \$arg1 \$arg2 ...
\end{syntax}
\begin{syntax}
\command{mmap} \$command \$matrix \$arg1 \$arg2 ...
\end{syntax}
\begin{syntax}
\command{rmap} \$command \$matrix \$arg1 \$arg2 ...
\end{syntax}
\begin{syntax}
\command{cmap} \$command \$matrix \$arg1 \$arg2 ...
\end{syntax}
\begin{syntax}
\command{vmap} \$command \$vector \$arg1 \$arg2 ...
\end{syntax}
\begin{args}
\$ndtype & Type of ND list. (e.g. 2D for matrix). \\
\$command & Command prefix to map over ND list. \\
\$ndlist & ND list to map with. \\
\$arg1 \$arg2 ... & Additional arguments to append to command call.
\end{args}

\begin{example}{Functional mapping}
\begin{lstlisting}
puts [vmap {format %.2f} {1 2 3}]; # Map a command prefix over a vector
puts [vmap max [transpose {{1 2 3} {4 5 6} {7 8 9}}]]; # Get vector of column maximums
puts [cmap max {{1 2 3} {4 5 6} {7 8 9}}]; # Shorthand way to get column maximums
namespace path ::tcl::mathfunc; # Makes all tcl math functions available as commands.
puts [vmap abs {-1 2 -3}]
\end{lstlisting}
\tcblower
\begin{lstlisting}
1.00 2.00 3.00
7 8 9
7 8 9
1 2 3
\end{lstlisting}
\end{example}
Note: the alias for column mapping actually performs a 1D map on the transpose of the matrix, so if performing multiple column maps, it is more efficient to transpose the matrix once and perform row mappings instead.
\clearpage
\subsection{ND Functional Mapping}
The command \cmdlink{nmap} is a general purpose mapping function for n-dimensional lists in Tcl. 
If multiple ND lists are provided for iteration, they must agree in dimension or be unity, like in \cmdlink{nset}. 
Returns an ND list in similar fashion to the Tcl \textit{lmap} command. 
Additionally, elements can be skipped with \textit{continue}, and the entire loop can be exited with \textit{break}.
\begin{syntax}
\command{nmap} \$nbody <\&> \\
nmap \$ndtype \$varName \$ndlist <\$varName \$ndlist ...> \$body <\&> 
\end{syntax}
\begin{args}
\$nbody & Body to evaluate, where ND object variables can be referenced with the ``@'' symbol. \\
\& & Option to return a temporary ND list object instead of the value. \\
\$ndtype & Type of ND list. (e.g. 2D for matrix). \\
\$varName & Variable name to iterate with. \\
\$ndlist & ND list to iterate over. \\
\$body & Body to evaluate at every iteration.
\end{args}
\subsubsection{Index Access}
The iteration indices of \cmdlink{nmap} are accessed with the commands \cmdlink{i}, \cmdlink{j}, \& \cmdlink{k}.
\begin{syntax}
\command{i} <\$axis>
\end{syntax}
\begin{args}	
\$axis & Dimension to access mapping index at. Default 0.
\end{args}
The commands \cmdlink{j} and \cmdlink{k} are simply shorthand for \cmdlink{i} with dimensions 1 and 2.
\begin{syntax}
\command{j}
\end{syntax}
\begin{syntax}
\command{k}
\end{syntax}
\clearpage
\subsection{Element-Wise Expressions}
The command \cmdlink{nexpr} performs element-wise expressions over multiple ND lists, using \cmdlink{nmap}. 
\begin{syntax}
\command{nexpr} \$nexpr <\&> \\
\command{nexpr} \$ndtype \$varName \$ndlist <\$varName \$ndlist ...> \$expr <\&>
\end{syntax}
\begin{args}
\$nexpr & Tcl expression, where ND object variables can be referenced with the ``@'' symbol. \\
\& & Option to return a temporary ND list object instead of the value. \\
\$ndtype & Type of ND list. (e.g. 2D for matrix). \\
\$varName & Variable name to iterate with. \\
\$ndlist & ND list to iterate over. \\
\$expr & Normal Tcl expression. \\
\end{args}
\begin{example}{Various uses of \cmdlink{nexpr}}
\begin{lstlisting}
set testmat {{1 2 3} {4 5 6} {7 8 9}}
# Simple negation
puts [nexpr 2D x $testmat {-$x}]
# Checkerboard
puts [nexpr 2D x $testmat {
    $x*([i]%2 + [j]%2 == 1?-1:1)
}]
# Addition with column vector 
puts [nexpr 2D x $testmat y {.1 .2 .3} {$x + $y}]
# Addition with row vector (using tcl::mathfunc::y)
puts [nexpr 2D x $testmat y {{.1 .2 .3}} {$x + $y}]
# Filter a vector using ``continue'' command (note, continue only continues at the lowest dimension).
set cutoff 3; # supports local variables in expr.
puts [nexpr 1D x {1 2 3 4 5 6} {$x > $cutoff ? [continue] : $x}]
\end{lstlisting}
\tcblower
\begin{lstlisting}
{-1 -2 -3} {-4 -5 -6} {-7 -8 -9}
{1 -2 3} {-4 5 -6} {7 -8 9}
{1.1 2.1 3.1} {4.2 5.2 6.2} {7.3 8.3 9.3}
{1.1 2.2 3.3} {4.1 5.2 6.3} {7.1 8.2 9.3}
1 2 3
\end{lstlisting}
\end{example}
\clearpage
\subsection{Element-Wise Operations}
If only performing a simple math operation with ND lists, the command \cmdlink{nop} can be used in lieu of \cmdlink{nexpr}. There are three ways to call \cmdlink{nop}, for single argument operations, operations with scalars, and element-wise operations. If performing element-wise operations, ND lists must be compatible in dimension just like in \cmdlink{nset} and \cmdlink{nexpr}. 

\begin{syntax}
\command{nop} \$ndobj \$op \$opargs <\&> \\
nop \$ndtype \$ndlist \$op \$opargs <\&>
\end{syntax}
\begin{args}
\$ndtype & Type of ND list. (e.g. 2D for matrix). \\
\$ndlist & ND list to perform element-wise operation over. \\
\$op & Math operator (using tcl::mathop namespace). \\
\$opargs & Operator arguments (see tcl::mathop documentation).
\end{args}

\begin{example}{Element-wise operations}
\begin{lstlisting}
puts [nop 1D {1 2 3} - {}]
puts [nop 1D {1 2 3} + 1]
\end{lstlisting}
\tcblower
\begin{lstlisting}
-1 -2 -3
2 3 4
\end{lstlisting}
\end{example}

\subsection{Stacking ND Lists}
ND lists can be stacked using the command \cmdlink{nstack}. 
Stacking combines ND lists along a specific axis. All other dimensions have to agree.
\begin{syntax}
\command{nstack} \$ndobj1 \$ndobj2 \$axis <\&> \\
nstack \$ndtype \$ndlist1 \$ndlist2 \$axis <\&> \\
\end{syntax}
\begin{args}
\$ndobj1 \$ndobj2 & ND list objects to stack (must be same dimensionality). \\
\$ndtype & Type of ND list. (e.g. 2D for matrix). \\
\$ndlist1 \$ndlist2 & ND lists to stack. \\
\$axis & Axis to stack along. E.g. 0 stacks along rows, 1 stacks along columns, etc. \\
\& & Option to return a temporary ND list object instead of the value.
\end{args}
\begin{example}{Combining matrices}
\begin{lstlisting}
puts [stack {{1 2}} {{3 4}}]
puts [augment {1 2} {3 4}]
\end{lstlisting}
\tcblower
\begin{lstlisting}
{1 2} {3 4}
{1 3} {2 4}
\end{lstlisting}
\end{example}
\clearpage
\end{document}
